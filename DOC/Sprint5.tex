Una vez implementada la funcionalidad de simulación, debe pensarse en añadir una persistencia al proyecto de los elementos que intervienen. Esto es necesario y primordial si se pretende crear una aplicación web para realizar simulaciones. Puesto que se creará una API usando Flask~\cite{Flask} que hará la función de servidor, se ha dedicido utilizar la herramienta para gestión de base de datos SQLAlchemy.
\subsection{Persistencia con SLQAlchemy}
SQLAlchemy~\cite{SqlAl} proporciona un kit de herramientas SQL que permiten manejar bases de datos de manera eficiente. Está formado por dos componentes:
\begin{itemize}
\item \textit{Core}: Es un conjunto de herramientas de SQL que da lugar a un nivel de abstracción sobre el mismo, mediante un lenguaje que utiliza expresiones generativas en Python para expresar órdenes SQL.
\item \textit{ORM}: Se trata de un asignador relacional de objetos, es decir, permite crear una base de datos de objetos virtuales que permite manipular la información de la base de datos, a priori incompatible, como objetos utilizable por un lenguaje de programación orientada a objetos.
\end{itemize}
Mediante las consultas basadas en funciones permite ejecutar las cláusulas SQL mediante funciones y expresiones en Python. Se pueden realizar numerosas acciones como subconsultas seleccionables, insertar, actualizar, eliminar o declarar un objeto, combinaciones internas y externas. El ORM permite almacenar en caché las colecciones y referencias de objetos una vez han sido cargados, dando lugar a que no sea necesario emitir SQL en cada acceso.\\

SQLAlchemy puede trabajar con bases de datos de SQLite, Postgresql, MySQL, Oracle, MS-SQL, Sybase y Firebird, entre otros.
\subsection{Modelos User y Home}
\subsection{Creación de un servidor con Flask}
\subsection{Desarrollo de la interfaz web}
