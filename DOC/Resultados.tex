\chapter{Resultados}
\label{cap:Resultados}
En este capítulo se explican los resultados obtenidos en cada una de las iteraciones en el desarrollo de este trabajo fin de grado y cómo se relacionan con los objetivos definidos.\\

Cada sección representa una iteración en el avance hacia el objetivo, donde a partir de la iteración 4 se realiza una integración continua hasta el hito final, pues dan como salida una versión funcional del objetivo. Es por esto que las historias de usuario anteriores a la iteración 4 tienen una prioridad alta.

\section{Identificación y adquisición de las variables del sistema}
\textbf{Historias de Usuario}\\

\begin{table}[H]
        \centering
        \begin{tabular}{|p{0.3\linewidth}|p{0.3\linewidth}|}
          \hline
          \multicolumn{2}{|c|}{Historia de usuario}\\ \hline
          \multicolumn{2}{|c|}{\textbf{Identificar las variables de entrada}}\\ \hline
          \textbf{Número:} 0 & \textbf{Prioridad:} Alta\\ \hline
          \textbf{Estimacion:} 1 día & \textbf{Iteracion:} 1\\ \hline
          \multicolumn{2}{|l|}{\textbf{Desarrollador responsable:} Pablo Palomino Gómez}\\ \hline
          \multicolumn{2}{|p{0.6\linewidth}|}{\textbf{Descripcion:} Identificar cuáles son las variables de entrada al sistema definido en el objetivo y sus dependencias}\\ \hline
        \end{tabular}
        \caption{Historia de usuario 0}
        \label{tab:hist0}
\end{table}
\begin{table}[H]
        \centering
        \begin{tabular}{|p{0.3\linewidth}|p{0.3\linewidth}|}
          \hline
          \multicolumn{2}{|c|}{Historia de usuario}\\ \hline
          \multicolumn{2}{|c|}{\textbf{Implementar módulos para el manejos de las APIs necesarias}}\\ \hline
          \textbf{Número:} 1 & \textbf{Prioridad:} Normal\\ \hline
          \textbf{Estimacion:} 10 días & \textbf{Iteracion:} 1\\ \hline
          \multicolumn{2}{|l|}{\textbf{Desarrollador responsable:} Pablo Palomino Gómez}\\ \hline
          \multicolumn{2}{|p{0.6\linewidth}|}{\textbf{Descripcion:} Se deben identificar las APIs de las que se solicitará información real necesaria para determinar los valores de las variables del sistema}\\ \hline
        \end{tabular}
        \caption{Historia de usuario 1}
        \label{tab:hist1}
\end{table}
\begin{table}[H]
        \centering
        \begin{tabular}{|p{0.3\linewidth}|p{0.3\linewidth}|}
          \hline
          \multicolumn{2}{|c|}{Historia de usuario}\\ \hline
          \multicolumn{2}{|c|}{\textbf{Identificar las variables de salida}}\\ \hline
          \textbf{Número:} 2 & \textbf{Prioridad:} Alta\\ \hline
          \textbf{Estimacion:} 1 día & \textbf{Iteracion:} 1\\ \hline
          \multicolumn{2}{|l|}{\textbf{Desarrollador responsable:} Pablo Palomino Gómez}\\ \hline
          \multicolumn{2}{|p{0.6\linewidth}|}{\textbf{Descripcion:} Identificar cuáles son las variables de salida del sistema definido en el objetivo y sus dependencias}\\ \hline
        \end{tabular}
        \caption{Historia de usuario 2}
        \label{tab:hist2}
\end{table}
\begin{table}[H]
        \centering
        \begin{tabular}{|p{0.3\linewidth}|p{0.3\linewidth}|}
          \hline
          \multicolumn{2}{|c|}{Historia de usuario}\\ \hline
          \multicolumn{2}{|c|}{\textbf{Identificar las variables de control}}\\ \hline
          \textbf{Número:} 3 & \textbf{Prioridad:} Alta\\ \hline
          \textbf{Estimacion:} 1 día & \textbf{Iteracion:} 1\\ \hline
          \multicolumn{2}{|l|}{\textbf{Desarrollador responsable:} Pablo Palomino Gómez}\\ \hline
          \multicolumn{2}{|p{0.6\linewidth}|}{\textbf{Descripcion:} Identificar cuáles son las variables de control del sistema definido en el objetivo y sus dependencias}\\ \hline
        \end{tabular}
        \caption{Historia de usuario 3}
        \label{tab:hist3}
\end{table}
\textbf{Desarrollo}\\

\subsection{Variables de entrada}
Son las fuentes que suministran energía al sistema. Habrá un total de tres:
\begin{itemize}
	\item \textbf{Energía fotovoltaica (EF)}\\ Energía procedente de las placas solares. Su valor viene determinado por varios factores, como son el número de módulos fotovoltaicos instalados y la máxima potencia que puede generar en cada momento, \gls{CMP}. Hace referencia a la potencia de salida, en watios que produce un panel fotovoltaico en condiciones de máxima iluminación solar, con una radiación de aproximadamente 1 kW/m2. Será dependiente de la situación meteorológica del momento. Como se puede observar, tendrá un valor máximo de obtención, que representa la cantidad de energía máxima que podemos obtener de los módulos fotovoltaicos en ese momento.
	\item \textbf{Energía de red (ER)}\\ Energía procedente de la compañía eléctrica como cliente particular. Al contrario que en el caso anterior, no existe un límite superior a la hora de obtener energía de esta fuente.
	\item \textbf{Energía almacenada en batería (EB)}\\ Energía obtenida de la batería de almacenaje. Al igual que la energía fotovoltaica tiene un límite superior y viene determinado por la cantidad de carga de la misma y la profundidad de descarga que se le puede realizar sin perjudicar su ciclo de vida y que debe ser de un 50\% como máximo.
\end{itemize}
\subsection{Adquisición de valores para las variables}
Como se ha mencionado, la energía fotovoltaica en una hora t será dependiente de la situación meteorológica en ese instante, algo evidente. Para contar con información meteorológica existen conjuntos de herramientas o servicios que ponen dicha información a disposición del desarrollador, como por ejemplo una API. Una \gls{API} es un conjunto de reglas o especificaciones que permite a las aplicaciones proporcionar servicios a otras o comunicarse. En este \gls{TFG} se emplea la \gls{API} oficial de AEMET \cite{Aemet}. Para su uso, se ha debido solicitar un \textbf{API key} ya que es una \gls{API} cerrada, esto es, su uso está restringido a un conjunto limitado de clientes. Para realizar una petición a la misma, debe incluirse el \textit{API key} mencionado anteriormente en la url solicitada, así como una serie de parámetros como el código del municipio que se desea consultar. Para obtener y procesar la información recibida por la \gls{API} se ha creado el módulo \textit{api\_aemet}, que contiene dos funciones para la obtención de información, una que se encarga de obtener la información referente al día en curso y otra que se encarga de obtener la información cuando la simulación se desea realizar de un día concreto. El motivo de diferenciarlas es que la \gls{API} no realiza una respuesta con predicciones por horas de un día distinto al actual, devolviendo en su defecto un texto en lenguaje natural con un resumen de lo ocurrido meteorológicamente dicho día para estos casos. Dichas funciones se explican a continuación.
\begin{itemize}
\item \textit{\textbf{get\_weather\_today}} (Listado~\ref{lst:aemet1}). Esta función realiza una petición a la \gls{API} mediante la librería \textit{requests}~\cite{Kenn11}, y en caso de obtener un código de éxito (código de estado http 200), procesa la respuesta recibida. Dicha respuesta es fácilmente procesable pues es un conjunto de parámetros clave-valor. Se procesa en la función \textbf{create\_weather\_buffer} y devuelve una lista con los 24 estados meteorológicos, correspondientes a las 24 horas de la simulación, del tipo ["Despejado", "Poco Nuboso", "Despejado", ..., "Despejado"].
\begin{lstlisting}[language=Python,float=ht,caption={Función para obtener los valores meteorológicos del día en curso},label={lst:aemet1}]
def get_weather_today(city):
    weather_buffer = []
    url = const.AEMET_URL_NOW.replace('$CITY', city)
    response = requests.get(url)
    data = response.json()

    if data['estado'] == 200:
        url = data['datos']
        response = requests.get(url)
        data = response.json()[0]
        weather_buffer = create_weather_buffer(data)
        return weather_buffer
    return None
\end{lstlisting}

\item \textit{\textbf{get\_weather\_archive}} (Listado~\ref{lst:aemet2}). Esta función realiza la petición a la \gls{API} de manera similar a la anterior, salvo que debe incluir en la url el parámetro específico de la fecha que se desea consultar. En este caso la respuesta no es directamente procesable pues no se trata de parámetros clave-valor. En su lugar se ha de procesar un texto en lenguaje natural (obsérvese en el Listado~\ref{lst:aemet1} como la respuesta es convertida a \textit{json} y en este caso es convertida a \textit{text}). Para su resolución, se ha implementado la función \textit{process\_weather\_archive} que realiza una \textbf{búsqueda de ocurrencias} de estados meteorológicos conocidos en el texto en lenguaje natural, obteniendo así información acerca del estado meterológico que se produjo ese día. Se forma el buffer con los estados meteorológicos encontrados en el texto y se devuelve una lista similar a la del caso anterior. En el Listado~\ref{lst:APIresponse2} se muestra un ejemplo del tipo de respuesta obtenida. En este caso se encontraría una ocurrencia de un estado meteorológico ('despejado'), por lo tanto se formaría el buffer del estado del cielo con 'Despejado'.
\begin{lstlisting}[language=Python,float=ht,caption={Función para obtener los valores meteorológicos de un día concreto},label={lst:aemet2}]
def get_weather_archive(date, city):
    weather_buffer = []
    province = city[:2]
    url = const.AEMET_URL_DATE.replace('$PROVINCE', province).replace('$DATE', date)

    response = requests.get(url)
    data = response.json()

    if data['estado'] == 200:
        url = data['datos']
        response = requests.get(url)
        raw_info = response.text
        weather_buffer = process_weather_archive(raw_info)
        return weather_buffer
    else:
        return None
\end{lstlisting}
\begin{lstlisting}[numbers=none,float=ht,caption={Ejemplo de respuesta de la API-AEMET para un día diferente al actual},label={lst:APIresponse2}]
AGENCIA ESTATAL DE METEOROLOGÍA

PREDICCIÓN PARA LA PROVINCIA DE TOLEDO
DÍA 12 DE FEBRERO DE 2019 A LAS 14:01 HORA OFICIAL
PREDICCIÓN VALIDA PARA EL MARTES 12


TOLEDO

Cielos despejados. Temperaturas mínimas en descenso. Temperaturas
máximas con pocos cambios predominando los aumentos en la Mancha.
Vientos flojos del este y nordeste tendiendo a flojos variables.
\end{lstlisting}
\end{itemize}

Las variables de entrada pueden tener simultaneamente valores distintos de 0, es decir, se puede obtener un tanto por ciento de la energía requerida de cada una de ellas. La selección de una u otra vendrá determinado por el precio en ese momento de cada una, ya que lo que se busca es minimizar el gasto producido. A continuación se muestran los cálculos que permiten determinar los precios asociados a las variables de entrada en una hora t: \\

	El precio de la energía fotovoltaica se calcula a partir de la inversión realizada en la instalación de los módulos fotovoltaicos y la cantidad de años en los que se desea amortizar dicha inversión. Así, el precio en €/Kw de EF se toma a partir de la Ecuación~\ref{eq:costoEF}.
	\begin{equation}
          \label{eq:costoEF}
	Costo_{EF} = \frac{coste_{anual}}{promedio^{kw}_{anual}} \textup{\euro}/kw
	\end{equation}
	Siendo el coste anual la cantidad invertida entre el número de años(n) en amortizarla, como se puede observar en la fórmula~\ref{eq:inversionEF}
	\begin{equation}
          \label{eq:inversionEF}
	Coste_{anual} = \frac{inversion}{n} \textup{\euro}
	\end{equation}


	El precio de la energía de red se obtiene haciendo uso de la \gls{API} oficial de \gls{REE} (e-sios) \cite{Ree}. Se ha debido solicitar un \textit{Token} de acceso que se utiliza en las llamadas a la misma al tratarse de una \gls{API} cerrada, análogamente al caso de la \gls{API} de AEMET. Para trabajar con esta \gls{API} se ha creado el módulo \textit{api\_esios}, que contiene la función \textit{get\_incoming\_prices}, la cuál se muestra en el Listado~\ref{lst:esios}.

	\begin{lstlisting}[language=Python,float=ht,caption={Función para obtener el precio del mercado eléctrico},label={lst:esios}]
	def get_incoming_prices(indicator, start, end):
	   url = const.ESIOS_URL.replace('$INDICATOR', indicator)
	   url = url.replace('$START_DATE', dt.datetime.strftime(start, '%Y/%m/%d'))
	   url = url.replace('$END_DATE', dt.datetime.strftime(end, '%Y/%m/%d'))

	   response = requests.get(url, headers=HEADERS)
	   if response.status_code == 200:
	      data = response.json()
	      price_buffer = create_price_buffer(data, start)
	      return price_buffer
	   return None
	\end{lstlisting}

	Esta función es llamada desde el proyecto con el indicador, que se corresponde con el precio que se desea consultar (en este caso \gls{PVPC}). Su código numérico es obtenido de las constantes del proyecto, al igual que la url necesaria para la petición (ESIOS\_URL), que se forma con los parámetros adecuados y así se realiza la petición \textit{get} haciendo uso de la librería \textit{requests}~\cite{Kenn11}.\\En este caso el \textit{\gls{API} key} no se concatena en la url, si no que debe incluirse en la cabecera de la petición en un campo específico, ya que se trata de una autenticación por token. Si la petición ha sido exitosa (código de respuesta http 200), la función retornará un \textit{buffer} de tamaño 24, que se corresponde con los valores del \gls{PVPC} en las 24 horas definidas en la simulación. Para ello llama a \textit{create\_price\_buffer}, que se encargará de generar la lista con los 24 valores del precio solicitado procesando la respuesta recibida de la petición a la \gls{API}.
	De esta manera se consigue el precio por Kw de la energía de red en cada momento.

	Por último, el precio de la energía para las baterías se puede calcular de un modo muy parecido al de la energía fotovoltaica. Hace referencia al coste que supone extraer energía almacenada en la batería y está relacionado con la inversión realizada en la batería (Ecuación~\ref{eq:costoEB}).
	\begin{equation}
          \label{eq:costoEB}
	Costo_{EB} = \frac{coste_{anual}}{capacidad_{bat}\cdot182,5} \textup{\euro}/kw
	\end{equation}
	Habiendo obtenido previamente el coste anual de forma similar a la fórmula~\ref{eq:inversionEF}. La constante 182,5 hace referencia al número de días del año (365) multiplicado por 0.5, debido a que no se va a realizar una profundidad de descarga mayor al 50\% de la capacidad total de la batería. El valor obtenido es el precio que supone extraer 1 Kw de la batería.
\subsection{Variables de salida}
Representan las fuentes de consumo de energía del sistema. Habrá un total de cuatro:
\begin{itemize}
\item \textbf{Consumo del hogar (C)}\\Demanda energética del hogar en cuestión, cuantía que debe ser satisfecha siempre, ya que es la energía que necesita el hogar para su uso cotidiano. En este \gls{TFG} se ha decidido trabajar con clientes de la empresa eléctrica Endesa S.A., pues el hogar del alumno es su cliente lo que permitirá trabajar con información real. Además cuenta con un área privada de cliente que permite acceder a datos analíticos del hogar (Figura~\ref{fig:endesa}) y permite descargar ficheros en formato de texto con el consumo por horas de un día determinado en el hogar del cliente, justo lo que se necesita para dotar de valor esta variable. Para adaptar la información del fichero al valor de la variable se ha creado la función \textit{read\_from\_file(filename)} del módulo \textit{client\_consumption} que devuelve una lista con los 24 consumos de las 24 horas de la simulación en KW, obtenidos del fichero proporcionado.
  \begin{figure}[H]
    \centering
    \includegraphics[width=10cm]{figs/Endesa.PNG}
    \caption{Panel de control del área de cliente de Endesa}
    \label{fig:endesa}
  \end{figure}
\item \textbf{Consumo interno del sistema ($ C_{int} $)}\\El sistema propuesto tiene un consumo constante de funcionamiento, cuyo valor se ha estimado en aproximadamente 2 Kw al día, alrededor de unos 0,088 watios por hora. Aquí se considera el consumo por funcionamiento de placas fotovoltaicas, realización de carga y descarga de batería.
	\item \textbf{Carga de batería (CB)}\\Cantidad de energía que se almacena en la batería para su posterior uso. Esta variable cobra sentido en el caso de un abaratamiento de alguna fuente de generación de energía, así se almacena para cuando el precio sea mayor.
	\item \textbf{Vertido al mercado eléctrico (CR)}\\Cantidad de energía que se vende al mercado eléctrico. Como particular, se puede disponer de una instalación fotovoltaica y verter energía a la red eléctrica, aunque es una práctica sujeta a numerosas trabas legales y dificultades en las que no se entrará en el desarrollo de este \gls{TFG}. Esta energía se vertería al intramercado de red conocido como el mercado SPOT, aquel donde los activos que se compran o venden se entregan al precio de mercado del instante de la compra/venta.
\end{itemize}

Como se puede observar, el vertido al mercado eléctrico tiene un beneficio económico que ha de tenerse en cuenta. Existe una retribución por Kw vertido a la red dependiente del momento del día, ya que como se ha comentado antes, el valor de compra/venta del mercado SPOT varía. Para la obtención de estos valores se vuelve a hacer uso de la ya mencionada \gls{API} e-sios, proporcionando a la función \textit{get\_incoming\_prices} el indicador del precio SPOT, presente en el fichero de constantes del proyecto. Análogo a la obtención del \gls{PVPC}, se retorna un buffer con los 24 valores requeridos del precio SPOT correspondientes a las 24 horas a simular.\\

Aunque a priori parezca que el hecho de cargar las baterías no tiene una compensación económica, esto no es del todo correcto. Existe un beneficio económico, aunque no directo, con esta práctica. Puede ser explicado como la cantidad ahorrada por almacenar esa energía y no consumirla, ya que se ha pagado por ella. Este valor puede verse como el mínimo de los precios de las fuentes de generación de energía en el momento de la carga. Veamos un breve ejemplo: En la hora t se ha obtenido energía fotovoltaica a un precio de 0,11 € el Kwh. Por otro lado, se ha obtenido energía de la compañía eléctrica contratada a un precio de 0,14 € el Kwh. El beneficio económico indirecto por cargar un Kw de energía en batería en esta hora t será de 0,11 €.

\subsection{Variables de control}
Existe otro conjunto de variables conocido como variables de control. Aunque se denotan como variables, en el caso concreto de una simulación son constantes, ya que sus valores están predefinidos para la simulación del modelo.\\
Este conjunto está formado por:
\begin{itemize}
	\item \textbf{Fecha de inicio}\\ Valor que hace referencia al inicio de la simulación. Este valor es representado mediante el módulo datetime de Python. Datetime~\cite{Dtpy} es un módulo de la librería estándar de Python que permite manipular y trabajar con fechas. Este valor será un día y una hora de ese día.
	\item \textbf{Fecha de fin}\\ Corresponde al fin de la simulación. Siempre será 24 horas a partir de la fecha de inicio. Al igual que el anterior, se representa haciendo uso de datetime.
	\item \textbf{Número de módulos fotovoltaicos}\\ El número de módulos fotovoltaicos juega un papel fundamental. A mayor número de módulos, se producirá mas energía, pero mayor deberá ser la inversión para adquirirlos.
	\item \textbf{Precio de un módulo fotovoltaico}\\ En este trabajo el tipo de módulo fotovoltaico será el que suele usar en domicilios particulares, con una potencia nominal en condiciones ideales de 50 watios. Este módulo tiene un precio por unidad de 40 €.
	\item \textbf{Años en amortizar la inversión de los módulos fotovoltaicos}\\ El número de años en los que se desea amortizar la inversión realizada en la adquisición de los módulos fotovoltaicos mediante su uso. Como se ha comentado anteriormente, no es algo trivial ya que determinará en gran medida el precio de extracción de energía fotovoltaica.
	\item \textbf{Precio de la batería}\\ En este trabajo el tipo de batería usado será una batería estacionaria compuesta por plomo abierto y gel. Este tipo de batería esta compuesta por dos vasos de 2V cada uno que disponen de un amplio rango de autonomía y una vida útil bastante larga, alrededor de unos 20 años. Son aconsejadas en instalaciones con un consumo medio (microondas, horno, lavadora, aire acondicionado, etc), es decir, perfectas para un hogar de tamaño normal. Como su tensión es de 2V, se debe instalar un total de 6 vasos en serie, al estar la instalación solar a 12V. Su precio es elevado debido a la gran capacidad, siendo éste 3900 €.
	\item \textbf{Capacidad de la batería}\\ El tipo de batería usado, es decir, batería estacionaria de 6 vasos, tiene una capacidad aproximada de 21 Kw. La profundidad de descarga de este tipo de batería es aproximadamente del 50\%, esto es, como se comento durante la explicación de las variables de entrada y salida, el tanto por ciento que se puede descargar dicha batería sin resultar perjudicial para su salud y por lo tanto afectar a su ciclo de vida útil.
	\item \textbf{Nivel de carga inicial de la batería}\\ Variable de control que define el estado de carga inicial de la batería a la hora de realizar la simulación del modelo.
	\item \textbf{Años en amortizar la inversión de la batería}\\ Como ocurre en el caso de la inversión fotovoltaica, se debe determinar el número de años en los que se desea realizar la amortización de la inversión por adquirir la batería. Al tratarse de un precio mucho más elevado debe ser mayor al del caso anterior, ya que si no se dispararía el precio de descargar las baterías y dejaría de ser una entrada a tener en cuenta al no resultar rentable.
\end{itemize}
Con esto quedan identificadas cada una de las variables que entran en juego en el modelo, así como su medio de adquisición.

\section{Aplicación de lógica difusa para la determinación de los estados meteorológicos}
\textbf{Historias de Usuario}\\

\begin{table}[H]
        \centering
        \begin{tabular}{|p{0.3\linewidth}|p{0.3\linewidth}|}
          \hline
          \multicolumn{2}{|c|}{Historia de usuario}\\ \hline
          \multicolumn{2}{|c|}{\textbf{Hallar los conjuntos difusos que se deben tratar}}\\ \hline
          \textbf{Número:} 4 & \textbf{Prioridad:} Alta\\ \hline
          \textbf{Estimacion:} 2 días & \textbf{Iteracion:} 2\\ \hline
          \multicolumn{2}{|l|}{\textbf{Desarrollador responsable:} Pablo Palomino Gómez}\\ \hline
          \multicolumn{2}{|p{0.6\linewidth}|}{\textbf{Descripcion:} Identificar los conjuntos difusos de estados meteorológicos}\\ \hline
        \end{tabular}
        \caption{Historia de usuario 4}
        \label{tab:hist4}
\end{table}
\begin{table}[H]
        \centering
        \begin{tabular}{|p{0.3\linewidth}|p{0.3\linewidth}|}
          \hline
          \multicolumn{2}{|c|}{Historia de usuario}\\ \hline
          \multicolumn{2}{|c|}{\textbf{Obtener un valor numérico de los estados meteorlógicos}}\\ \hline
          \textbf{Número:} 5 & \textbf{Prioridad:} Alta\\ \hline
          \textbf{Estimacion:} 2 días & \textbf{Iteracion:} 2\\ \hline
          \multicolumn{2}{|l|}{\textbf{Desarrollador responsable:} Pablo Palomino Gómez}\\ \hline
          \multicolumn{2}{|p{0.6\linewidth}|}{\textbf{Descripcion:} Dotar de valor discreto los conjuntos difusos mediante el cálculo de centroides}\\ \hline
        \end{tabular}
        \caption{Historia de usuario 5}
        \label{tab:hist5}
\end{table}
\textbf{Desarrollo}\\

Un ejemplo de respuesta en lenguaje natural se vio en el Listado~\ref{lst:APIresponse2}. En el Listado~\ref{lst:APIresponse1} se muestra (simplificada) la respuesta recibida en el caso de la consulta del día en curso (\textcopyright AEMET).
\begin{lstlisting}[numbers=none,float=ht,caption={Ejemplo de respuesta de la API-AEMET para el día en curso},label={lst:APIresponse1}]
{
 origen: {
	productor: "Agencia Estatal de Meteorología - AEMET. Gobierno de España",
	web: "http://www.aemet.es",
	language: "es",
	copyright: "AEMET. Autorizado el uso de la información y su reproducción citando a AEMET como autora de la misma.",
	notaLegal: "http://www.aemet.es/es/nota_legal"
 },
 elaborado: "2019-2-12",
 nombre: "Consuegra",
 provincia: "Toledo",
 prediccion: {
 	dia: [
		{
		 estadoCielo: [
					{
					 periodo: "08",
					 descripcion: "Cubierto"
					},
					{
					 periodo: "09",
					 descripcion: "Cubierto con lluvia escasa"
					},
					{
					 periodo: "10",
					 descripcion: "Cubierto con lluvia escasa"
					},
					{
					 periodo: "11",
					 descripcion: "Cubierto"
					},
					...
		 ]
		}
	}
}
\end{lstlisting}
\\

Cómo se puede observar se diferencia de la respuesta en texto del Listado~\ref{lst:APIresponse2}, pues se trata de una respuesta \gls{JSON}, que es un formato de texto simple que se utiliza para el intercambio de información, aunque tuvo sus inicios en Javascript, haciendo honor a su nombre (\textit{Javascript Object Notation}).\\
En el campo 'predicción' existe un subcampo 'estadoCielo' (entre otros que se han obviado por no ser de interés en este trabajo) que contiene una lista con los estados meteorológicos de la previsión. Cada elemento de la lista contiene dos valores: periodo (hora del día de esa previsión) y descripción (cadena de texto que describe el estado del cielo, mismo conjunto de palabras que se emplea para encontrar las ocurrencia en el texto en el otro tipo de respuesta). Claramente existe un problema con los conjuntos de estados meteorológicos que se obtienen de procesar las peticiones a la \gls{API} \gls{AEMET}, ya que para la determinación de la máxima energía fotovoltaica posible en una hora t debe conocerse \gls{CMP}, la potencia nominal posible (potencia que es capaz de suministrar el módulo fotovoltaico), directamente proporcional al estado meteorológico, el cuál es un texto que describe la situación y no un valor numérico que representa los watios que puede dar un módulo en esas condiciones, y a priori no se dispone de una forma directa de relacionarlas. Por tanto, se emplea \textbf{lógica difusa} para resolver la problemática mencionada anteriormente.
\subsection{Lógica difusa}
La teoría de la lógica difusa proporciona un marco matemático que permite modelar la incertidumbre de los procesos cognitivos humanos para poder ser tratable por un computador. Estos procesos cognitivos hacen referencia a expresiones del tipo:
\begin{itemize}
	\item Si no vives \textit{lejos} puedes ir en bicicleta.
	\item Si hace \textit{mucho} frío llévate un chaquetón.
\end{itemize}
Los humanos son capaces de interpretar estos valores rápidamente. Sin embargo, las máquinas no tienen esa capacidad, debido a que no existe un valor cuantitativo que indique la distancia a la que se refiere la palabra \textit{lejos} o que temperatura es \textit{mucho} frío. Si se intentan trasladar estas reglas a código, aparecen dificultades ya que no se puede procesar numéricamente. Una opción es definir intervalos de valores que comprenderá cada palabra (por ejemplo, tomando \textit{lejos} como la distancia comprendida entre 5 y 10 kilómetros), pero esto no es preciso ya que para un computador, la distancia de 5,01 kilómetros sería igual de lejos que 9,9 kilómetros, cuando en realidad la interpretación correcta no es así. Con esto queda a la vista que la lógica convencional no trata de forma eficiente este problema presentando numerosas limitaciones. Otro ejemplo típico es el mostrado en la Figura~\ref{fig:ejemplo_logica}, donde se puede observar como la lógica clásica interpretaría erróneamente el hecho: \textit{Una persona de dos metros es alta}, pues clasificaría una persona de 1,99 metros como no alta, mientras que la lógica difusa lo clasificaría mediante un grado de pertenencia. La solución pasa por emplear un método de razonamiento afín a la lógica difusa.\\
\begin{figure}[!h]
	\centering
	\includegraphics[width=9cm]{figs/tipos_logica.png}
	\caption{Lógica clásica vs. lógica difusa}
        \label{fig:ejemplo_logica}
\end{figure}

La lógica difusa~\cite{Morc11} permite representar matemáticamente la \textbf{incertidumbre}. Según Zadeh~\cite{Zad73}, "\textit{Cuando aumenta la complejidad, los enunciados precisos pierden su significado y los enunciados útiles pierden precisión.}", es decir, \textit{los árboles no te dejan ver el bosque}, pues prácticamente cualquier problema del mundo puede resolverse partiendo de unas variables de entrada y buscando obtener como objetivo un conjunto de variables de salida. La lógica difusa establece esta relación entre variables de forma correcta.

\subsection{Conjuntos difusos}
En la teoría de conjuntos de la lógica clásica, el grado de pertenencia puede tomar solo los valores 0 y 1, que representan que el elemento pertenece o no pertenece al conjunto. En la lógica difusa existe el concepto de \textbf{conjunto difuso}~\cite{Zad65}, establecido por Zadeh. Para trabajar con valores difusos se realiza un proceso denominado \textit{fuzzificación} que da resultados difusos. Estos resultados se someten a un proceso de \textit{defuzzificación} para transformarse en valores discretos (llamados \textit{crisp}), que tendrán un \textbf{grado de pertenecia} a los conjuntos difusos el cuál será un valor en el intervalo [0, 1], y representa cuanto pertenece al conjunto.\\

Así pues, hay un claro ejemplo de conjuntos difusos con los estados meteorológicos. En la Figura~\ref{fig:estadoCielo} se muestran los estados meteorológicos que proporciona la \gls{API} de \gls{AEMET}~\cite{Aemet} (Obtenida de la web oficial de \textcopyright AEMET).
\begin{figure}[H]
	\centering
	\includegraphics[width=17cm]{figs/estadoCieloAEMET.png}
	\caption{Posibles estados del cielo en AEMET}
	\label{fig:estadoCielo}
\end{figure}
Como se puede observar existe un gran número de estados meteorológicos, y en muchos de ellos un módulo fotovoltaico no genera energía. En estos estados el valor máximo a generar por los módulos será de 0 watios. Los estados favorables (donde un módulo genera una potencia mayor a 0 watios) serán representados como conjuntos difusos. En la Tabla~\ref{tab:estadosFavorables} se muestran cuáles son estos estados. Por ejemplo, no se puede determinar cuantos watios se producen como máximo con un tiempo \textit{Despejado} o \textit{Cubierto con nubes altas}, pero podemos mostrar los conjuntos difusos y gráficamente comprobar su distribución, para obtener un valor discreto de cada conjunto difuso, denominado \textbf{centroide} o centro de gravedad del conjunto difuso. Este proceso es conocido como razonamiento aproximado a partir de inferencia difusa. La inferencia difusa permite obtener un valor de salida para un valor de entrada empleando la teoría de conjuntos difusos. Un ejemplo de cuestión a resolver es: \textit{¿Qué potencia pico se puede obtener de un módulo fotovoltaico si hay intervalos nubosos?}. Cómo hemos comentado anteriormente, la respuesta sería el centroide del conjunto difuso \textit{Intervalos Nubosos}. El centroide es el punto que divide el conjunto difuso en dos partes de igual masa. En la Ecuación~\ref{eq:centroide} se muestra el procedimiento para calcularlo, realizando el sumatorio de las potencias tomadas por su grado de pertenencia al conjunto dividido entre el sumatorio de dichos grados de pertenencia. En la Tabla~\ref{tab:estadosFavorables} se incluye el valor del centroide de cada etiqueta lingüistica.\\
\begin{equation}
        \label{eq:centroide}
        Centroide = \frac{\sum_{x=i}^{n} x \mu_{A}(x)}{\sum_{x=i}^{n} \mu_{A}(x)}
\end{equation}
Para representar las variables lingüisticas, se parte del hipotético caso en el que cada variable desciende de manera ligeramente más inclinada que asciende, representando el factor de adaptación de un módulo fotovoltaico a un nuevo estado meteorológico. En la Figura~\ref{fig:fuzzySets} se pueden observar las variables lingüisticas definidas en la Tabla~\ref{tab:estadosFavorables}. En el eje horizontal se encuentra la potencia máxima que puede generar el módulo fotovoltaico en watios y en el eje vertical el grado de pertenencia a cada variable en el intervalo [0,1]. Si se avanza en el eje horizontal, se observa que el estado meteorológico es cambiante hacia estados favorables, y viceversa, ya que esto es directamente proporcional a la máxima potencia posible generada por el módulo.

Cada conjunto difuso es una función PI o \textbf{trapezoidal}, ya que no existe un único punto donde el grado de pertenencia al conjunto es 1, si no que se mantiene ese valor de pertenencia hasta que el módulo comienza a experimentar un cambio en el estado del cielo y se debe adaptar a dicho estado.

\begin{figure}[h]
	\centering
	\includegraphics[width=17cm]{figs/Fuzzy_diagram.pdf}
	\caption{Conjuntos difusos de los estados meteorológicos}
        \label{fig:fuzzySets}
\end{figure}

Tal y cómo se comentó en la iteración anterior, se utilizan módulos fotovoltaicos con \gls{CMP} de 50 watios, que solo podría ser alcanzada en el estado meteorológico óptimo (\textit{Despejado}), y en función de este dato, se toman los valores para obtener los centroides.
\begin{table}[H]
        \centering
        \begin{tabular}{|c|l|c|}
                \hline
                \textbf{Etiqueta lingüistica} & \textbf{Descripción} & \textbf{Centroide} \\ \hline
                e10 & Despejado & 48 W \\ \hline
                e9 & Poco nuboso & 43,16 W \\ \hline
                e8 & Nubes altas & 38,16 W \\ \hline
                e7 & Intervalos nubosos & 33,16 W \\ \hline
                e6 & Intervalos nubosos con lluvia escasa & 28,16 W \\ \hline
                e5 & Intervalos nubosos con lluvia & 23,16 W \\ \hline
                e4 & Nuboso & 18,16 W \\ \hline
                e3 & Nuboso con lluvia escasa & 13,16 W \\ \hline
                e2 & Cubierto & 8,16 W \\ \hline
                e1 & Cubierto con lluvia escasa & 2,66 W\\ \hline
        \end{tabular}
        \caption{Variable lingüistica de la CMP}
        \label{tab:estadosFavorables}
\end{table}

El dominio de la variable lingüistica es [0, 50] Watios. Estos valores se almacenarán en un diccionario, disponible en el fichero de constantes del proyecto (módulo \textit{project\_constants}). Dicho diccionario será usado para realizar el parseo de los estados meteorológicos obtenidos de la \gls{API} \gls{AEMET} (cadenas de texto) a valores cuantitativos (centroide del conjunto difuso), y poder ser usables por el sistema para determinar la \textbf{máxima energía fotovoltaica} que se puede obtener en un momento determinado.

\section{Creación de las relaciones y restricciones propias del modelo}
\textbf{Historias de Usuario}\\
\begin{table}[H]
        \centering
        \begin{tabular}{|p{0.3\linewidth}|p{0.3\linewidth}|}
          \hline
          \multicolumn{2}{|c|}{Historia de usuario}\\ \hline
          \multicolumn{2}{|c|}{\textbf{Identificar las restricciones y función objetivo del PSR}}\\ \hline
          \textbf{Número:} 6 & \textbf{Prioridad:} Alta\\ \hline
          \textbf{Estimacion:} 3 días & \textbf{Iteracion:} 3\\ \hline
          \multicolumn{2}{|l|}{\textbf{Desarrollador responsable:} Pablo Palomino Gómez}\\ \hline
          \multicolumn{2}{|p{0.6\linewidth}|}{\textbf{Descripcion:} Determinar el dominio de cada variable e identificar las restricciones del PSR}\\ \hline
        \end{tabular}
        \caption{Historia de usuario 6}
        \label{tab:hist6}
\end{table}
\begin{table}[H]
        \centering
        \begin{tabular}{|p{0.3\linewidth}|p{0.3\linewidth}|}
          \hline
          \multicolumn{2}{|c|}{Historia de usuario}\\ \hline
          \multicolumn{2}{|c|}{\textbf{Implementar la clase Simulation }}\\ \hline
          \textbf{Número:} 7 & \textbf{Prioridad:} Alta\\ \hline
          \textbf{Estimacion:} 5 días & \textbf{Iteracion:} 3\\ \hline
          \multicolumn{2}{|l|}{\textbf{Desarrollador responsable:} Pablo Palomino Gómez}\\ \hline
          \multicolumn{2}{|p{0.6\linewidth}|}{\textbf{Descripcion:} Implementar la clase de la que se instanciarán objetos que representarán simulaciones. Debe estudiarse la información que ha de contener un objeto Simulation}\\ \hline
        \end{tabular}
        \caption{Historia de usuario 7}
        \label{tab:hist7}
\end{table}
\textbf{Desarrollo}\\

Cómo se comento en el capítulo relativo al Objetivo del trabajo fin de grado, éste se aborda como un \textbf{problema de satisfacción de restricciones}.\\

La programación por restricciones es una metodología software que permite resolver problemas de gran complejidad, típicamente NP. Ésta metodología ha generado mucha espectación en el área de la inteligencia artificial desde la década de los 60, ya que tiene un gran potencial para la resolución de problemas reales. La idea básica de la programación por restricciones es primero declarar una serie de restricciones sobre el dominio del problema que atañe, para después dar con soluciones que satisfacen las anteriores restricciones de la forma más optima posible. Así, un problema de satisfacción de restricciones~\cite{Russ06} está caracterizado por:
\begin{itemize}
	\item Un conjunto de variables, donde cada variable dispone de un dominio de valores que puede tomar.
	\item Un conjunto de restricciones, que permite conocer las posibles combinaciones de las variables.
	\item La solución al PSR será la asignación de valores a las variables de forma que se satisfacen las restricciones y se alcanza el objetivo, representado típicamente como una función a optimizar.
\end{itemize}
Las restricciones se caracterizan por su \textbf{aridad}, que viene a ser el número de variables que involucra. Pudiendo ser unarias, si solo involucran una variable; binarias, si involucran dos variables; y n-arias, si involucran más de dos variables. Se deben tener en cuenta un tipo de restricción adicional, ya que están presentes en este trabajo, como son las \textbf{restricciones lineales}, expresadas teóricamente de la forma~\ref{eq:rest_lineal}
\begin{equation}
  \label{eq:rest_lineal}
  \sum_{i}^{n} a_{i}x_{i} (<,\leq,=,\geq,>,\neq) c
\end{equation}
siendo a el coeficiente de la variable x y c constante.\\

\subsection{Variables del PSR}

En los hitos 1 y 2 se identificaron las variables del problema, las cuáles están divididas en tres grupos: variables de entrada, variables de salida y variables de control. Sin embargo conviene mencionar cuáles de estas variables son las propias del problema de satisfacción de restricciones, y cuáles formarán las restricciones. El objetivo del problema es obtener valores de energía para los paneles fotovoltaicos, baterías y red eléctrica, de tal modo que se cubra la demanda energética del hogar y que el gasto económico sea el menor. Es por esto que las variables propias del problema de satisfacción de restricciones serán:
\begin{itemize}
\item \textbf{Energía fotovoltaica (EF)}. Energía obtenida de los módulos fotovoltaicos.~\ref{eq:dom_ef}
\begin{equation}
        \label{eq:dom_ef}
        Dom_{EF} = [0, EF_{max}]
\end{equation}
\item \textbf{Energía de red (ER)}. Energía importada de la red eléctrica.~\ref{eq:dom_er}
\begin{equation}
        \label{eq:dom_er}
        Dom_{ER} = [0, +\infty)
\end{equation}
\item \textbf{Energía de batería (EB)}. Energía obtenida de la batería.~\ref{eq:dom_eb}
\begin{equation}
        \label{eq:dom_eb}
        Dom_{EB} = [0, +\infty)
\end{equation}
\item \textbf{Consumo de batería (CB)}. Energía consumida para cargar la batería.~\ref{eq:dom_cb}
\begin{equation}
        \label{eq:dom_cb}
        Dom_{CB} = [0, +\infty)
\end{equation}
\item \textbf{Consumo de red (CR)}. Energía vertida a red a cambio de retribución económica.~\ref{eq:dom_cr}
\begin{equation}
        \label{eq:dom_cr}
        Dom_{CR} = [0, +\infty)
\end{equation}
\end{itemize}

Las restricciones estarán definidas en función de dichas variables y cada solución al problema estará formada por un valor para cada una de estas variables. Éstos valores satisfacen las restricciones y además serán los óptimos para que se produzca el menor gasto económico posible. El resto de variables (variables de control) determinarán las propias restricciones y el valor de las anteriores dependerá de éstas en una hora concreta t, entre 0 y 24h.\\

\subsection{Restricciones del PSR}
A continuación se determinan las restricciones a las que está sometido el modelo en una hora t:
\begin{itemize}
\item \textbf{Toda la energía generada debe ser consumida}.~\ref{eq:restr1}\\ \\Hace referencia al principio básico de la energía, la energía que se produce se consume de un modo u otro, no es posible que la suma de las variables correspondientes a la generación de energía (EB, ER y EF) sea distinta a la suma de las variables que hacen referencia al consumo de energía (CR, CB, $ C_{int} $ y C). Ésto debe producirse en cada una de las horas de la simulación. Así, tenemos una restricción lineal y n-aria correspondiente a la suma de las 24 horas correspondientes a una simulación, por lo que ésta restricciones a efectos prácticos es tomada como 24 restricciones a cumplir.
\begin{equation}
        \label{eq:restr1}
        \sum_{i=0}^{23} EF_{i}+ER_{i}+EB_{i} = CR_{i}+CB_{i}+C_{int}+C
\end{equation}

\item \textbf{No se puede producir energía fotovoltaica durante la noche}.~\ref{eq:restr2}\\ \\Algo obvio, pues sin luz solar la energía fotovoltaica no es posible. Ésto no está controlado en la API AEMET, ya que las peticiones relativas a la noche no reflejan una descripción propia del tiempo nocturo, si no que devuelve los mismos valores independientemente de si existe luz solar, por lo que debe manejarse mediante una restricción. Para este trabajo las horas de la noche serán las pertenecientes al intervalo temporal desde las 22:00 pm hasta las 7:00 am. Cómo posible trabajo futuro, podría determinarse este intervalo en función de la estación del año para que pueda ser un intervalo con mayor grado de efectividad. Se trata de una restricción unaria, donde para ciertos valores de t, EF debe ser 0. Es por esto que esta restricción a efectos prácticos es tomada como 9 restricciones (las 9 horas de noche definidas anteriormente)
\begin{equation}
        \label{eq:restr2}
        EF_{noche} = 0
\end{equation}

\item \textbf{La energía fotovoltaica generada no puede ser mayor que la máxima energía fotovoltaica en t}.~\ref{eq:restr3}\\ \\No se puede superar el umbral de generación de energía fotovoltaica establecido por la potencia nominal máxima de esa hora t, pues se estaría violando la capacidad real de producción de los módulos fotovoltaicos del sistema. Es una restricción lineal unaria, ya que la energía fotovoltaica máxima de cada hora t es constante, pues como se comentó anteriormente, sólo es dependiente del número de módulos fotovoltaicos y la situación meteorológica (obtenida de la API AEMET). A efectos prácticos, esta restricción es tomada como 24 restricciones a cumplir referentes a las 24 horas de la simulación.
\begin{equation}
        \label{eq:restr3}
        \sum_{i=0}^{23} EF_{i} \leq EF_{i}^{max}
\end{equation}

\item \textbf{La energía obtenida de la batería no puede ser mayor que el nivel de batería actual teniendo en cuenta la profundidad máxima de descarga}.~\ref{eq:restr4}\\ \\Básicamente no se puede obtener una cantidad de energía mayor a la posible en esa hora t, que vendrá determinada por la diferencia entre el  nivel de carga disponible al comienzo de esa hora y la capacidad máxima de la batería por la profundidad de descarga (50\%), para evitar daños en su ciclo de vida útil. Restricción unaria, pues solo involucra la variable EB, ya que el resto de elementos de la restricción son constantes en una hora t (nivel de carga actual, capacidad máxima de la batería y profundida de descarga). Al igual que las restricciones anteriores es lineal y a efectos prácticos representa 24 restricciones a cumplir.
\begin{equation}
        \label{eq:restr4}
        \sum_{i=0}^{23} EB_{i} \leq nivel_{i-1} - capacidad_{max} * profundidad_{descarga}
\end{equation}

\item \textbf{El consumo para cargar la batería no puede ser mayor que la capacidad de la misma menos el nivel restante después de t}.~\ref{eq:restr5}\\Parecido a la restricción anterior, en esta se modela el hecho de cargar la batería (CB) en cada hora t, el cuál está condicionado por la cantidad de batería restante para completar la carga (100\%), obtenido mediante la diferencia entre la capacidad máxima de la misma y lo consumido en la hora t (nivel de carga antes de comenzar la hora t menos la energía consumida de batería en t). Restricción binaria pues involucra tanto el consumo de batería (CB) como la energía de batería (EB), siendo la capacidad máxima de la batería y el nivel de carga en t constantes. Es tomado como 24 restricciones ya que debe cumplirse en cada una de las 24 horas de una simulación.
\begin{equation}
        \label{eq:restr5}
        \sum_{i=0}^{23} CB_{i} \leq capacidad_{max}- (nivel_{i-1} - EB_{i})
\end{equation}

\end{itemize}

Por lo tanto, a efectos prácticos, el PSR tiene 81 restricciones que satisfacer para determinar los valores de las variables.

\subsection{Función Objetivo}
Cómo se ha comentado antes, un problema de satisfacción de restricciones está determinado por un conjunto de variables y sus dominios, un conjuntos de restricciones de esas variables, y un objetivo. Por el momento se dispone de los dos primeros, por que en este apartado se procede a determinar el último, \textbf{la función objetivo}.\\

Un PSR que cuenta únicamente con variables y restricciones para esas variables podrá tener numerosas soluciones, representadas como una tupla con valores para cada variable. Añadiendo un objetivo al PSR se consigue unificar la solución, pues de todas esas soluciones, sólo una optimizará un objetivo concreto, y contendrá los valores óptimos de cada variable para ello.\\

Como se ha comentado a lo largo del desarrollo de este trabajo fin de grado, el objetivo del problema es \textbf{minimizar el gasto económico} producido mediante la optimización de energía en cada simulación de 24 horas, por lo tanto se buscarán valores de las variables que, además de satisfacer el conjunto de restricciones, sean óptimos para que el gasto económico sea mínimo. Éste gasto económico es dependiente del precio en la hora t de cada una de las energías que representan las variables. De estos precios se habló y se implementó la forma de obtenerlos en el hito 1. Sus valores son constantes en cada hora t. Dicho esto, la función objetivo a minimizar está formada por el sumatorio de los gastos económico en cada hora t, por lo que representa el gasto económico de toda la simulación.~\ref{eq:funcionObjetivo}
\begin{equation}
\label{eq:funcionObjetivo}
f(x) = \sum_{i=0}^{23} EF_{i}P_{i}^{F} + ER_{i}P_{i}^{R_{PVPC}} + EB_{i}P_{i}^{B_{out}} - CB_{i}P_{i}^{B_{in}} - CR_{i}P_{i}^{R_{SPOT}}
\end{equation}
Siendo:
\begin{itemize}
\item $ EF_{i}P_{i}^{F} $: Gasto económico producido al generar energía fotovoltaica en la hora t.
\item $ ER_{i}P_{i}^{R_{PVPC}} $: Gasto económico producido al importar energía a la compañía eléctrica en la hora t.
\item $ EB_{i}P_{i}^{B_{out}} $: Gasto económico producido al descargar la batería en la hora t.
\item $ CB_{i}P_{i}^{B_{in}} $: Ganancia económica producida al cargar la batería en la hora t.
\item $ CR_{i}P_{i}^{R_{SPOT}} $: Ganancia económica producida al verter energía al mercado eléctrico en la hora t.
\end{itemize}

En cada simulación se buscará que el valor de $ f(x) $ sea el menor posible y la solución al PSR de esa simulación será los valores que deben tomar las variables para hacerlo posible.

\subsection{Implementación de la clase Simulation}
Vistos los apartados anteriores, es la hora de implementar una clase para representar el modelo de simulación, y poder crear objetos que representan una simulación concreta. Esta clase se llama \textbf{Simulation}~\ref{fig:simulation} y está disponible en el módulo simulation.\\

Los atributos de clase de Simulation serán las variables de control que tendrá cada objeto correspondiente a una simulación:
\begin{itemize}
        \item start: variable de control referente a fecha de inicio de la simulación. Es de tipo datetime.
        \item end: variable de control referente a fecha de fin de la simulación. Al igual que start, es una variable de formato fecha (datetime).
        \item ef\_price: variable de control referente al precio de generar energía fotovoltaica. Se trata de un número tipo float, pues tendrá el mismo valor en toda la simulación (se obtiene a partir de la inversión realizada)
        \item er\_price: variable de control referente a los precios que tendrá el PVPC en cada hora de la simulación. Es una lista de 24 elementos de tipo float.
        \item eb\_price: variable de control referente al precio de descargar la batería. Es un número de tipo float ya que será igual en las 24 horas de la simulación.
        \item cr\_price: variable de control referente a los precios SPOT en cada hora de la simulación, es decir, el precio del vertido de energía a la red eléctrica.
        \item cb\_price: variable de control referente a los precios de cargar la batería. Cómo es dependiente de los precios de energía fotovoltaica y de red de cada hora t, se trata de una lista con 24 valores.
        \item battery\_capacity: variable de control referente a la capacidad total de la batería. Variable de tipo float.
        \item battery\_level: variable de control que representa el nivel inicial de carga de la batería. Variable de tipo float.
        \item discharge\_depth: variable de control referente a la profundidad de descarga permitida en la batería. Variable de tipo float.
        \item max\_ef\_buffer: Más que una variable de control, representa los 24 valores máximos posible de energía fotovoltaica, obtenidos como se comentó anteriormente mediante la información de la API AEMET y el número de módulos fotovoltaicos, por ende, se trata de una lista de 24 valores.
        \item c\_int: referente a la variable de control del consumo interno del sistema. Variable de tipo float.
        \item c: referente al consumo del hogar, lista de los 24 valores con el consumo del hogar en las 24 horas de la simulación.

Las funciones de esta clase sirven para calcular algunos de los atributos anteriores. En la figura~\ref{fig:simulation} se muestra la clase UML de Simulation.
\end{itemize}

\begin{figure}[H]
        \centering
        \includegraphics[width=6cm]{figs/simulation_class.png}
        \caption{Clase Simulation}
        \label{fig:simulation}
\end{figure}

En el próximo hito se implementará cada una de las restricciones para permitir ejecutar la simulación.

\section{Generación optimizada de energía mediante programación lineal}
\textbf{Historias de Usuario}\\

\begin{table}[H]
        \centering
        \begin{tabular}{|p{0.3\linewidth}|p{0.3\linewidth}|}
          \hline
          \multicolumn{2}{|c|}{Historia de usuario}\\ \hline
          \multicolumn{2}{|c|}{\textbf{Hallar un medio de optimización del PSR}}\\ \hline
          \textbf{Número:} 8 & \textbf{Prioridad:} Alta\\ \hline
          \textbf{Estimacion:} 5 días & \textbf{Iteracion:} 4\\ \hline
          \multicolumn{2}{|l|}{\textbf{Desarrollador responsable:} Pablo Palomino Gómez}\\ \hline
          \multicolumn{2}{|p{0.6\linewidth}|}{\textbf{Descripcion:} Investigar sobre un medio de optimización de un objetivo sujeto a un número muy grande de restricciones}\\ \hline
        \end{tabular}
        \caption{Historia de usuario 8}
        \label{tab:hist8}
\end{table}
\begin{table}[H]
        \centering
        \begin{tabular}{|p{0.3\linewidth}|p{0.3\linewidth}|}
          \hline
          \multicolumn{2}{|c|}{Historia de usuario}\\ \hline
          \multicolumn{2}{|c|}{\textbf{Implementar las restricciones identificadas anteriormente}}\\ \hline
          \textbf{Número:} 9 & \textbf{Prioridad:} Alta\\ \hline
          \textbf{Estimacion:} 7 días & \textbf{Iteracion:} 4\\ \hline
          \multicolumn{2}{|l|}{\textbf{Desarrollador responsable:} Pablo Palomino Gómez}\\ \hline
          \multicolumn{2}{|p{0.6\linewidth}|}{\textbf{Descripcion:} Implementar en la clase Simulation las restricciones que se identificaron en la iteración anterior para poder ser utilizadas en la optimización}\\ \hline
        \end{tabular}
        \caption{Historia de usuario 9}
        \label{tab:hist9}
\end{table}
\begin{table}[H]
        \centering
        \begin{tabular}{|p{0.3\linewidth}|p{0.3\linewidth}|}
          \hline
          \multicolumn{2}{|c|}{Historia de usuario}\\ \hline
          \multicolumn{2}{|c|}{\textbf{Implementar la optimización y realizar una simulación}}\\ \hline
          \textbf{Número:} 10 & \textbf{Prioridad:} Alta\\ \hline
          \textbf{Estimacion:}10 días & \textbf{Iteracion:} 4\\ \hline
          \multicolumn{2}{|l|}{\textbf{Desarrollador responsable:} Pablo Palomino Gómez}\\ \hline
          \multicolumn{2}{|p{0.6\linewidth}|}{\textbf{Descripcion:} Implementar un método en la clase simulation llamado \textit{optimize()} que realice la optimización a través de las variables de clase, restricciones y valores necesarios, y procesar el resultado en un fichero de reporte con la información asociada}\\ \hline
        \end{tabular}
        \caption{Historia de usuario 10}
        \label{tab:hist10}
\end{table}
\textbf{Desarrollo}\\

La programación lineal~\cite{Loom64} tiene como objetivo optimizar una función lineal cuyas variables están sujetas a un conjunto de restricciones lineales.
Se trata de un campo de la matemática muy efectivo para la resolución de este tipo de problemas. En el ámbito de las ciencias de la computación existen librerías que permiten emplear algoritmos de programación lineal para la resolución de problemas de optimización. En este \gls{TFG} se empleará \textbf{SciPy}, un ecosistema de librerías de código abierto con numerosas herramientas para matemáticas, ciencia e ingeniería.

\subsection{Optimización con SciPy}
Scipy~\cite{Scip} proporciona un conjunto de paquetes de computación científica para el lenguaje Python como son Numpy, Scipy, Matplotlib, iPython, SymPy y Pandas. En este caso el trabajo se centra en el módulo Scipy.optimize~\cite{SciOp}, que contiene las herramientas de Scipy para optimización. Proporciona numerosos algoritmos de optimización para uso común:
\begin{itemize}
\item Minimización sin restricciones y restringida de funciones escalares multivariadas.
\item Optimización global mediante fuerza bruta.
\item Minimización de mínimos cuadrados.
\item Minimización de funciones univariantes escalares y búsqueda de soluciones.
\item Solución de sistemas de ecuaciones multivariables con una gran cantidad de algoritmos.
\end{itemize}
Para el caso propio de este \gls{TFG} en el que el objetivo es minimizar una función sujeta a un gran conjuntos de restricciones, lo más conveniente es hacer uso del módulo \textbf{linprog} de Scipy.optimize, específico para trabajar con programación lineal. Resuelve problemas del tipo definido en el Listado~\ref{lst:linprog}, donde:
\begin{itemize}
\item A\_ub representa los coeficientes de las restricciones definidas como inecuaciones.
\item b\_ub representa las constantes del tipo de restricción inecuación.
\item A\_eq representa los coeficientes de las restricciones de igualdad, es decir, ecuaciones.
\item b\_eq representa las constantes del tipo de restricción de igualdad.
\item (lb, ub) representan los límites inferior y superior del dominio de la variable x.
\item c es la función a minimizar, dependiente de la variable x.
\end{itemize}
\begin{lstlisting}[language=Python,float=ht,numbers=none,caption={Tipo de problema aplicable a Scipy.optimize.linprog},label={lst:linprog}]
  # Minimizar:
  c @ x
  # Sujeto a:
  A_ub @ x <= b_ub
  A_eq @ x == b_eq
  lb <= x <= ub
\end{lstlisting}
El caso particular de este trabajo se adapta perfectamente a dicho modelo de problema. Pero antes de emplear el algoritmo linprog, se deben implementar cada una de las restricciones del modelo, algo complejo en este caso pues existen numerosas restricciones al tratarse de una función lineal, pues cada una de las variables involucradas en la función a minimizar tendrá realmente 24 valores, correspondientes a las 24 horas del dia que se desea optimizar, y desde el punto de vista de la implementación, será tenido en cuenta como 24 variables distintas.\\

Antes de implementar cada una de las restricciones, se debe hacer una modificación en la clase Simulation. Se añaden cinco nuevos atributos a la clase:
\begin{itemize}
\item \textbf{f}: Esta variable representa la función objetivo (véase la Ecuación~\ref{eq:funcionObjetivo}), expresada como una lista con los coeficientes de cada variable en la función, los cuales representan el precio del tipo de energía asociado a la variable. Al tratarse de un sumatorio de 24 iteraciones y la expresión estar formada por 5 variables, esta lista contendrá 120 elementos resultantes de la suma entre los 24 valores de cada una de las variables. Para dotar de valores a la lista se ha implementado la función \textit{generate\_function\_coeficients()}, que mediante 24 iteraciones concatena a la lista el valor correspondiente de cada coeficiente de variable, que se encuentran en los atributos de clase definidos en la iteración 3 (Véase la representación UML de Simulation en la Figura~\ref{fig:simulation}). Esta variable se corresponde con \textit{c} en el modelo de problema para Scipy Linprog del Listado~\ref{lst:linprog}.
\item \textbf{A\_ub, b\_ub}: Como se comentó anteriormente representan las restricciones del tipo inecuación. En A\_ub se almacena una lista con los coeficientes de las inecuaciones en una lista por restricción, de tal modo que se tiene una lista de listas (lista de dos dimensiones) del tipo: [[coeficientes restr. 1], [coeficientes restr.2], ...]. En b\_ub se tiene una lista con las constantes de las inecuaciones, por lo que el tamaño de b\_ub y A\_ub debe ser igual para que se realice el \textit{matching} de coeficientes con constantes por inecuación.
\item \textbf{A\_eq, b\_eq}: Similar a las dos listas anteriores, pero en este caso se trata de las restricciones de igualdad. Las listas tienen el mismo formato.
\end{itemize}
Estas variables serán primordiales a la hora de ejecutar el algoritmo linprog pues de sus valores serán dependientes los resultados para cada variable. Definidas las variables que contendrán los valores de las restricciones se pasará a continuación a la implementación de dichas restricciones.
\subsection{Implementación de las restricciones del tipo R1}
La restricción R1 se corresponde con que toda la energía generada debe ser consumida (Véase la Ecuación~\ref{eq:restr1}). Se trata de una restricción de igualdad, por lo que debe dotarse de valor a A\_eq y b\_eq. Es una restricción lineal por lo que desde el punto de vista de la implementación se traduce en 24 restricciones, una por hora de la simulación. En el Listado~\ref{lst:restr1} se muestra la función \textit{generate\_restriction\_1()} que realiza dicha tarea.\\
\begin{lstlisting}[language=Python,float=ht,caption={Restricciones del tipo R1},label={lst:restr1}]
def generate_restriction_1(self):
    for i in range(0, 24):
        restr_coef = [0]*5*24
        restr_coef[i*5] = 1
        restr_coef[i*5+1] = 1
        restr_coef[i*5+2] = 1
        restr_coef[i*5+3] = -1
        restr_coef[i*5+4] = -1
        self.A_eq.append(restr_coef)
        self.b_eq.append(self.c_int + self.c[i])
\end{lstlisting}

Primero se deben mostrar a la izquierda de la restricción las variables y a la derecha las constantes. En A\_eq se debe concatenar una lista por restricción del sumatorio que contendrá los valores 0 o 1 en función de la condición mostrada en el Listado~\ref{lst:coef}.\\

\begin{lstlisting}[numbers=none,float=ht,caption={Condición para dotar de valor los coeficientes},label={lst:coef}]
  Si la variable de esa posición aparece en la restricción
      restr_coef[posicion] = 1
  Si no
      restr_coef[posicion] = 0
\end{lstlisting}
Como se puede observar en el Listado~\ref{lst:restr1}, por cada iteración de las 24 (correspondientes a las 24 horas de la simulación) primero se crea una lista \textbf{restr\_coef} con solo valores 0. Esta lista dispone de 120 elementos, pues la restricción es realmente el sumatorio de 24 restricciones y existen 5 variables en la expresión (EF, ER, EB, CR y CB). C y $ C_{int}$ no se contemplan en la expresión ya que son constantes en cada hora. Como resultado se obtendrá en A\_eq 24 listas de 120 elementos cada una, de los cuales todos toman el valor 0 excepto los relativos a las posiciones de las variables que entran en juego en la restricción de esa iteración. Es primordial que se preserve el ordenamiento de las variables en todas las restricciones. Deben tener el mismo orden que en la función objetivo y tomar 1 si aparecen o 0 si no (Podrán tomar el valor -1 si van precedidas de una resta en la restricción). En el caso de b\_eq, se concatenan 24 valores, uno por iteración, correspondientes al valor constante de la restricción de esa iteración. Obsérvese cómo se realizaría la primera iteración, correspondiente a la hora 0 de la simulación:\\

\textit{Deben asignarse con 1 $ EF_{0} $, $ ER_{0} $ y $ EB_{0} $, pues su coeficiente en la restricción es +1. Deben asignarse con -1 $ CR_{0} $ y $ CB_{0} $, pues su coeficiente es -1 en la restricción. El resto de elementos de la lista deben ser 0 (correspondientes al resto de coeficientes de variables para t= 1, 2, 3, ..., 23). Esta lista se concatena en A\_eq. En b\_eq se concatena el valor constante de esta restricción, que es $ c_{int} + C_{t}^{prop} $. Con esto queda implementada la restricción de tipo 1 correspondiente a la hora 0 de la simulación.}

\subsection{Implementación de las restricciones del tipo R2}
La restricción R2 hace referencia a que no se puede producir energía fotovoltaica durante la noche (Véase la ecuación~\ref{eq:restr2}). En este \gls{TFG} se definen estos valores como los comprendidos entre las 22:00 y las 7:00. Como se observa en el Listado~\ref{lst:restr2}, para generar las restricciones de este tipo, en cada iteración se inicializa la lista de 120 valores con ceros de manera análoga a las restricciones de tipo R1. Después, para determinar la hora real correspondiente de la iteración en curso, se debe sumar a la hora de inicio de la simulación el número de iteración actual. El uso del módulo de la librería estándar de Python \textit{datetime}~\cite{Dtpy} hace que sea posible manejar variables en formato hora o fecha. En las iteraciones en las que la hora actual esté comprendida en las definidas como horas nocturnas, el valor de la posición de EF (energía fotovoltaica) en esa iteración tomará el valor 1. Estas listas resultantes de cada iteración se van concatenando con A\_eq, pues son restricciones de igualdad. En cuanto a b\_eq, por cada iteración se concatena un 0, pues el valor constante de esta restricción es 0 debido a que la generación de energía fotovoltaica de noche es nula. Tras la ejecución de la función \textit{generate\_restriction\_2()}, A\_eq cuenta con 24 listas más, que son las 24 restricciones del tipo R2.
\begin{lstlisting}[language=Python,float=ht,caption={Restricciones del tipo R2},label={lst:restr2}]
def generate_restriction_2(self):
    for i in range(0, 24):
        restr_coef = [0]*5*24
        time = (self.start+dt.timedelta(hours=i)).time()
        if time >= dt.time(21, 30) or time <= dt.time(7, 00):
            restr_coef[i*5] = 1
        self.A_eq.append(restr_coef)
        self.b_eq.append(0)
\end{lstlisting}
\subsection{Implementación de las restricciones del tipo R3}
Las restricciones del tipo R3 hacen que se cumpla que la energía fotovoltaica generada no puede ser mayor que la máxima energía fotovoltaica en t, siendo t cada hora de la simulación (Véase la ecuación~\ref{eq:restr3}). Se trata de una restricción de tipo inecuación, por lo que en este caso deberán concatenarse sus valores a A\_ub y b\_ub. En la variable de clase \textit{self.max\_ef\_buffer} se dispone de una lista con los 24 valores correspondientes a la energía fotovoltaica máxima de cada hora de la simulación. Cada elemento de esta lista representa la parte constante de cada restricción de este tipo, por ello, en cuanto a b\_ub se refiere, basta con concatenar \textit{self.max\_ef\_buffer}. En el caso de la parte de variables (A\_ub), al igual que en los casos anteriores se realizan 24 iteraciones correspondientes a las 24 horas de la simulación, y en cada una de ellas, la lista de 120 elementos toma el valor 1 únicamente en la posición relativa a la energía fotovoltaica, pues es la única que entra en juego en este tipo de restricción. La lista generada en cada iteración se concatena con el resto de restricciones en A\_ub.
\begin{lstlisting}[language=Python,float=ht,caption={Restricciones del tipo R3},label={lst:restr3}]
def generate_restriction_3(self):
    for i in range(0, 24):
        restr_coef = [0]*5*24
        restr_coef[i*5] = 1
        self.A_ub.append(restr_coef)
    self.b_ub.extend(self.max_ef_buffer)
\end{lstlisting}
\subsection{Implementación de las restricciones del tipo R4}
Las restricciones de tipo R4 hacen que se cumpla que la energía obtenida de la batería no puede ser mayor que el nivel de batería actual teniendo en cuenta la profundidad máxima de descarga (Véase la Ecuación~\ref{eq:restr4}). Son restricciones de tipo inecuación por lo que deben modificarse A\_ub y b\_ub. En este caso, la restricción correspondiente a la hora 0 de la simulación debe separarse de las restantes, pues en ese punto la cantidad de carga de la batería se obtiene directamente de la variable de clase que contiene el nivel inicial de batería (\textit{self.battery\_level}) (Ecuación~\ref{eq:restr4t1}) y en el resto de casos se obtiene mediante un conjunto de operaciones (Ecuación~\ref{eq:restr4t2}). Esto permite calcular el nivel actual de batería en la hora i a partir de la que hubo inicialmente, mediante el sumatorio de las cargas y descargas que se han realizado desde que comenzó la simulación. En el Listado~\ref{lst:restr4} se puede observar la función \textit{generate\_restriction\_4()}, encargada de la implementación de las restricciones de tipo R4 comprendidas entre las horas 1 y 24 de la simulación. Por cada iteración, en la lista de coeficientes se coloca un 1 en la posición relativa a EB, pues es la dependiente de esta restricción. Después, se realizan iteraciones desde 0 hasta la iteración anterior a la actual, para comprobar el nivel actual de batería, posicionando los valores 1 en EB y -1 en CB. Cuando la lista de coeficientes está completa en esa iteración, se añade a A\_ub, y en b\_ub se concatena la parte constante de este tipo de restricción, que viene a ser la diferencia entre el nivel inicial de batería y la capacidad de la misma por su profundidad de descarga.
\begin{equation}
  \label{eq:restr4t1}
  EB_{0} \leq initial\_level - capacity * depth
\end{equation}
\begin{equation}
  \label{eq:restr4t2}
  EB_{t} \leq initial\_level + \sum_{i=0}^{t-1}(-EB_{i}+CB_{i}) - capacity * depth
\end{equation}
\begin{lstlisting}[language=Python,float=ht,caption={Restricciones del tipo R4},label={lst:restr4}]
def generate_restriction_4(self):
    for i in range(1, 24):
        restr_coef = [0]*5*24
        restr_coef[i*5+2] = 1
        for j in range(0, i-1):
            restr_coef[j*5+2] = 1
            restr_coef[j*5+4] = -1
        self.A_ub.append(restr_coef)
        self.b_ub.append(self.battery_level
            -self.battery_capacity*self.discharge_depth)
\end{lstlisting}
\subsection{Implementación de las restricciones del tipo R5}
Las restricciones de tipo R5 se encargan de que el consumo para cargar la batería no pueda ser mayor que la capacidad de la misma menos el nivel restante después de t (Véase la Ecuación~\ref{eq:restr5}). Las restricciones de este tipo son muy parecidas a las de tipo R4, con la diferencia de que las retricciones de tipo R4 se encargan de regular la energía que se descarga de la batería y las restricciones de tipo R5 regulan la energía que se carga a la batería. La hora 0 de la simulación debe implementarse aparte análogamente al tipo anterior, pues el nivel actual de batería se determina en función de las cargas y descargas que se han producido desde que comenzó el día que se desea simular. En este caso la restricción de la hora 0 es muy sencilla pues tras agrupar a la izquierda de la inecuación las variables y a la derecha las constantes y ordenar las variables preservando el orden de \textit{f} se obtiene la restricción de la Ecuación~\ref{eq:restr5t1}. Para implementar esta restricción simplemente se debe dar valor de -1 a la posición relativa a $ EB_{0} $ y 1 a la posición relativa a $ CB_{0} $, para después añadir a sus respectivas listas la lista de coeficientes y el valor constante de la restricción. Para el resto de restricciones de este tipo (hora 1 a 24) se usa la función \textit{generate\_restriccion\_5()} cuya traza es similar a \textit{generate\_restriccion\_4()} exceptuando los valores que toman las posiciones relativas a las variables dependientes de la restricción.
\begin{equation}
  \label{eq:restr5t1}
  CB_{0} - EB_{0} <= capacity - initial\_level
  -EB_{0} + CB_{0} \leq capacity - initial\_level
\end{equation}
\begin{lstlisting}[language=Python,float=ht,caption={Restricciones del tipo R5},label={lst:restr5}]
def generate_restriction_5(self):
    for i in range(1, 24):
        restr_coef = [0]*5*24
        restr_coef[i*5+4] = 1
        restr_coef[i*5+2] = -1
        for j in range(0, i-1):
            restr_coef[j*5+2] = -1
            restr_coef[j*5+4] = 1
        self.A_ub.append(restr_coef)
        self.b_ub.append(self.battery_capacity - self.battery_level)
\end{lstlisting}
\subsection{Generación optimizada de energía}
Una vez implementadas todas las restricciones necesarias del PSR es la hora de implementar el algoritmo linprog de Scipy. El método \textit{optimize()} de la clase Simulation se encarga de esta tarea. En ella deben llamarse a todos los métodos encargados de las restricciones, para así poder contener en A\_eq, b\_eq, A\_ub y b\_ub los datos de variables y constantes necesarios. Después deben determinarse los dominios que pueden tomar las variables, definidos en la iteración anterior. Finalmente se efectúa el algoritmo linprog sobre todos los datos y se obtiene como respuesta un conjunto de valores que han de ser interpretados, para lo que se añaden a la clase Simulation las siguientes funciones auxiliares:
\begin{itemize}
\item \textbf{store\_result(result)}: se encarga de almacenar en un fichero los resultados obtenidos, indicando fecha de simulación, gasto económico producido y cantidad de energía de cada fuente de entrada y salida por horas. Esta información es almacenada en un fichero llamado \textit{simulation\_fecha.txt}, que sirve como reporte de la simulación. Para obtener cada valor se itera sobre la lista de valores en bruto \textbf{\textit{res.x.to\_list()}} separando cada valor de variable en su iteración y variable correspondiente.
\item \textbf{prepare\_result(result)}: se encarga de procesar una salida a la simulación alternativa a la anterior, pues retorna los resultados utilizando el formato \gls{JSON}. Ésto será útil cuando se haga una petición de simulación desde el servidor y deba devolverse el resultado en este formato para poder ser procesado fácilmente. Se utiliza el método \textit{json.dumps()} para generar el objeto \gls{JSON} a partir de un diccionario clave-valor (Véase el Listado~\ref{lst:resultJson}). La función \textit{self.prepare\_hours(values)} procesa la lista de valores de variables en bruto a un diccionario clave-valor.
\begin{lstlisting}[language=Python,float=ht,caption={Función de procesamiento del resultado a formato json},label={lst:resultJson}]
def prepare_result(self, res):
    values = res.x.tolist()
    data = {
      "start" : self.start.strftime("%Y-%m-%d %H:%M:%S"),
      "end" : self.end.strftime("%Y-%m-%d %H:%M:%S"),
      "result" : res.fun,
      "hours" : self.prepare_hours(values)
    }

    return json.dumps(data)
\end{lstlisting}
\end{itemize}
Puesto que ya se cuenta con el esqueleto del proceso para llevar a cabo una simulación, se procede a realizar un caso de prueba para el lunes día 11 de Marzo de 2019.
\subsection{Caso de prueba: Simulación del 11 de Marzo}
Tal como se explicó en la Sección~\ref{sec:hito1}, en el Área Cliente de Endesa es posible obtener el consumo realizado por horas de un día determinado. Esto va a permitir conocer cual fue el consumo del 11 de Marzo y, gracias al trabajo realizado con la \gls{API} e-sios~\cite{Ree}, saber cual fue la cuantía económica del consumo total de ese día. Se podrá comparar con el gasto económico que se obtendrá de la simulación.\\

El 11 de marzo en el hogar del alumno se produjo un consumo total de 7 KWh. El \gls{PVPC} por defecto de ese día tuvo un valor medio de 0,12 €, por lo tanto como mínimo el consumo del día fue de 0,84 €. El fichero de texto que contiene el consumo diario de Endesa sigue el formato mostrado en el Listado~\ref{lst:11marzo} (\textcopyright Endesa S.A.).
\begin{lstlisting}[float=ht,numbers=none,caption={Fichero de consumo por horas de Endesa},label={lst:11marzo}]
CUPS:				ESXXXXXXXXXXXXXXXXXXXX
Fecha :				11/03/2019
Fecha y hora de extracción :	23/03/2019 10:57:52
Tarifa :			No se encontró la tarifa
  Fecha 			   Hora 			    Consumo (Wh)
2019-03-11			00:00-01:00				110.0
2019-03-11			01:00-02:00				80.0
2019-03-11			02:00-03:00				166.0
2019-03-11			03:00-04:00				141.0
2019-03-11			04:00-05:00				95.0
2019-03-11			05:00-06:00				126.0
2019-03-11			06:00-07:00				186.0
2019-03-11			07:00-08:00				217.0
2019-03-11			08:00-09:00				568.0
2019-03-11			09:00-10:00				692.0
2019-03-11			10:00-11:00				280.0
2019-03-11			11:00-12:00				216.0
2019-03-11			12:00-13:00				149.0
2019-03-11			13:00-14:00				348.0
2019-03-11			14:00-15:00				677.0
2019-03-11			15:00-16:00				339.0
2019-03-11			16:00-17:00				368.0
2019-03-11			17:00-18:00				192.0
2019-03-11			18:00-19:00				202.0
2019-03-11			19:00-20:00				175.0
2019-03-11			20:00-21:00				648.0
2019-03-11			21:00-22:00				479.0
2019-03-11			22:00-23:00				318.0
2019-03-11			23:00-00:00				270.0

Total (Wh):				7042.0
\end{lstlisting}
\\
Mediante la función \textit{read\_from\_file()} del módulo \textit{client\_consumption} se procesa el fichero de consumo, se establecen los consumos en la unidad de KWh y se retorna una lista con los 24 consumos de las 24 horas del día. Esta lista será el valor que toma la variable de Simulation \textit{self.c}. Debido a que aún no se dispone de interfaz ni servidor, para ejecutar la simulación se ha empleado el intérprete de comandos ipython3. Se debe crear un objeto simulación con los argumentos necesarios indicados a lo largo de este capítulo y que se pueden observar en la clase UML de Simulation (Figura~\ref{fig:simulation}). Tras instanciar en un objeto la simulación, se debe invocar el método \textit{optimize()} que implementa las restricciones y ejecuta el algoritmo de programación lineal, obteniendo como salida el contenido del Listado~\ref{lst:output}.
\begin{lstlisting}[language=bash,float=ht,numbers=none,caption={Salida del algoritmo \textit{linprog}},label={lst:output}]
fun: 0.6187091110614097
message: 'Optimization terminated successfully.'
nit: 236
slack: array([ 8.16000000e-01,  8.16000000e-01,  8.16000000e-01,
        8.16000000e-01,  2.66000000e-01,  2.66000000e-01,
        2.66000000e-01,  2.66000000e-01,  0.00000000e+00,
                                      . . .
        5.91880000e+00,  4.68600000e+00,  1.05000000e+01,
        5.02480000e+00,  1.05000000e+01,  1.05000000e+01])
status: 0
success: True
x: array([0.00000000e+00, 1.98800000e-01, 0.00000000e+00,
       0.00000000e+00, 0.00000000e+00, 0.00000000e+00,
       1.68800000e-01, 0.00000000e+00, 0.00000000e+00,
                                     . . .
       0.00000000e+00, 0.00000000e+00, 0.00000000e+00,
       3.58800000e-01, 0.00000000e+00, 0.00000000e+00])
\end{lstlisting}
\\ \\ \\

Esta información se interpreta de la siguiente forma:
\begin{itemize}
\item \textit{fun} es el resultado que toma la función \textit{f(x)} (Función~\ref{eq:funcionObjetivo}) como resultado de asignar a las variables los valores de la optimización. Dicho valor representa el \textbf{gasto económico} producido para el día simulado. Como se puede observar, se trata de 0,61 euros frente a los 0,84 euros que supuso el consumo del día utilizando solamente la energía de la compañia eléctrica. Se ha producido un ahorro del 27.38\% lo cuál es bastante significativo, teniendo en cuenta que el valor medio del \gls{PVPC} del día 11 de marzo fue relativamente bajo con respecto al valor que suele oscilar (0,15 euros).
\item \textit{message} devuelve \textit{feedback} sobre si la optimización ha tenido o no éxito, tomando en los campos \textit{status} y \textit{success} los valores 0 y True en caso de éxito; y los valores 1 y False en caso de error.
\item \textit{nit} hace referencia a las iteraciones realizadas sobre las variables para determinar el valor óptimo obtenido. En este caso se han realizado 236 iteraciones.
\item \textit{slack} contiene un array con los valores de las variables de holgura, donde cada variable de holgura se corresponde con una de las restricciones de desigualdad.
\item \textit{x} contiene el array en bruto de valores de variables que han hecho posible la optimización. Esta información ha de ser tratada y clasificada para poder ser interpretada, por lo que mediante la función \textit{store\_result()} se ha generado un fichero de reporte de simulación que contiene el valor de cada variable (\gls{EF}, \gls{ER}, \gls{EB}, \gls{CB}, \gls{CR}, C y $ C_{int} $) en cada una de las horas de la simulación. Estos valores representan la cantidad de energía en KWh que debe tomarse en cada variable para producir la minimización del gasto económico.
\end{itemize}
El Anexo \ref{cap:AnexoA} contiene el Listado~\ref{lst:simulationReport} con la información procesada de la simulación. Cómo se puede observar, en la hora de comienzo de la simulación (0:00), la única fuente de energía es \gls{ER}, pues una de las restricciones implementadas es que durante la noche la captación de energía fotovoltaica no es posible, y la batería aún no se ha cargado para permitir su extracción. Se puede observar que en algunas horas se genera más energía de la necesaria para satisfacer el consumo (C + $ C_{int} $). Esto es debido a que mediante la \gls{API} \gls{AEMET}~\cite{Ree} se comprueba que en ese momento el precio \gls{PVPC} se encuentra relativamente bajo por lo que resulta rentable importar más energía de la necesaria y almacenarla en batería para una hora en la que los precios suban. Es el caso de lo ocurrido en la hora 2:00 por ejemplo. A partir del amacener comienza a tenerse en cuenta \gls{EF}, es el caso de la hora 8:00, donde se aprovecha su potencial para importar mucha más energía de la requerida y se almacena (Se requieren 0.65 KWh pero sin embargo son generados 0,816 KWh, aprovechando la diferencia para cargar la batería). Esto permite que en horas donde por causas meteorológicas no sea posible emplear \gls{EF} se tengan alternativas a \gls{ER}, o también durante la noche, como es el caso de la hora 11:00, cuando \gls{EF} no es posible y \gls{ER} tiene un precio más alto, se obtenga la totalidad de la energía requerida de la batería (\gls{EB}).\\

Tras esta simulación y la interpretación de resultados se han obtenido las siguientes conclusiones:
\begin{itemize}
\item Se ha producido un ahorro económico en la simulación del día 11 de marzo mediante el caso propuesto con respecto al caso real, a pesar que es un día donde el \gls{PVPC} se encuentra relativamente bajo, por lo que el ahorro producido en un caso de simulación de la media habría sido mucho mayor.
\item Mediante el empleo de distintas fuentes de energía se ha alcanzado lo que puede considerarse como eficiencia energética.
  \item La actuación reactiva del sistema con respecto a la información cambiante ya sea meteorológica o del mercado eléctrico permite garantizar que en cada momento la cantidad de energía en cada una de las fuentes es la cantidad óptima.
\end{itemize}

\section{Persistencia de datos y creación de la aplicación web}
\textbf{Historias de Usuario}\\

\begin{table}[H]
        \centering
        \begin{tabular}{|p{0.3\linewidth}|p{0.3\linewidth}|}
          \hline
          \multicolumn{2}{|c|}{Historia de usuario}\\ \hline
          \multicolumn{2}{|c|}{\textbf{Investigar y elegir un \textit{framework} de base de datos para Python}}\\ \hline
          \textbf{Número:} 11 & \textbf{Prioridad:} Media\\ \hline
          \textbf{Estimacion:}2 días & \textbf{Iteracion:} 5\\ \hline
          \multicolumn{2}{|l|}{\textbf{Desarrollador responsable:} Pablo Palomino Gómez}\\ \hline
          \multicolumn{2}{|p{0.6\linewidth}|}{\textbf{Descripcion:} Estudiar y elegir un marco de trabajo para el uso de bases de datos SQL en el lenguaje Python}\\ \hline
        \end{tabular}
        \caption{Historia de usuario 11}
        \label{tab:hist11}
\end{table}
\begin{table}[H]
        \centering
        \begin{tabular}{|p{0.3\linewidth}|p{0.3\linewidth}|}
          \hline
          \multicolumn{2}{|c|}{Historia de usuario}\\ \hline
          \multicolumn{2}{|c|}{\textbf{Implementar los modelos de la base de datos}}\\ \hline
          \textbf{Número:} 12 & \textbf{Prioridad:} Media\\ \hline
          \textbf{Estimacion:}10 días & \textbf{Iteracion:} 5\\ \hline
          \multicolumn{2}{|l|}{\textbf{Desarrollador responsable:} Pablo Palomino Gómez}\\ \hline
          \multicolumn{2}{|p{0.6\linewidth}|}{\textbf{Descripcion:} Se deben definir las tablas que contendrá la base de datos. Implementar los modelos de dichas tablas, crear la base de datos con las tablas y realizar los tests necesarios}\\ \hline
        \end{tabular}
        \caption{Historia de usuario 12}
        \label{tab:hist12}
\end{table}
\begin{table}[H]
        \centering
        \begin{tabular}{|p{0.3\linewidth}|p{0.3\linewidth}|}
          \hline
          \multicolumn{2}{|c|}{Historia de usuario}\\ \hline
          \multicolumn{2}{|c|}{\textbf{Implementar las rutas y lógica de la aplicación Flask}}\\ \hline
          \textbf{Número:} 13 & \textbf{Prioridad:} Media\\ \hline
          \textbf{Estimacion:}10 días & \textbf{Iteracion:} 5\\ \hline
          \multicolumn{2}{|l|}{\textbf{Desarrollador responsable:} Pablo Palomino Gómez}\\ \hline
          \multicolumn{2}{|p{0.6\linewidth}|}{\textbf{Descripcion:} El objetivo es identificar los \textit{endpoints} que contendrá la aplicación Flask e implementar la lógica necesaria para realizar una simulación mediante una petición al \textit{endpoint} pertinente}\\ \hline
        \end{tabular}
        \caption{Historia de usuario 13}
        \label{tab:hist13}
\end{table}
\begin{table}[H]
        \centering
        \begin{tabular}{|p{0.3\linewidth}|p{0.3\linewidth}|}
          \hline
          \multicolumn{2}{|c|}{Historia de usuario}\\ \hline
          \multicolumn{2}{|c|}{\textbf{Implementar la autenticación en la aplicación}}\\ \hline
          \textbf{Número:} 14 & \textbf{Prioridad:} Media\\ \hline
          \textbf{Estimacion:}7 días & \textbf{Iteracion:} 5\\ \hline
          \multicolumn{2}{|l|}{\textbf{Desarrollador responsable:} Pablo Palomino Gómez}\\ \hline
          \multicolumn{2}{|p{0.6\linewidth}|}{\textbf{Descripcion:} Hallar un medio de autenticación de usuarios y control de sesiones en la aplicación e implementar la lógica necesaria para autenticarse mediante el acceso al \textit{endpoint} pertinente}\\ \hline
        \end{tabular}
        \caption{Historia de usuario 14}
        \label{tab:hist14}
\end{table}
\begin{table}[H]
        \centering
        \begin{tabular}{|p{0.3\linewidth}|p{0.3\linewidth}|}
          \hline
          \multicolumn{2}{|c|}{Historia de usuario}\\ \hline
          \multicolumn{2}{|c|}{\textbf{Implementarla vista de inicio de sesión}}\\ \hline
          \textbf{Número:} 15 & \textbf{Prioridad:} Media\\ \hline
          \textbf{Estimacion:}5 días & \textbf{Iteracion:} 5\\ \hline
          \multicolumn{2}{|l|}{\textbf{Desarrollador responsable:} Pablo Palomino Gómez}\\ \hline
          \multicolumn{2}{|p{0.6\linewidth}|}{\textbf{Descripcion:} Debe crearse la vista html para inicio de sesión o creación de una cuenta y enlazarla con el endpoint correspondiente en la aplicación Flask así como los tests necesarios}\\ \hline
        \end{tabular}
        \caption{Historia de usuario 15}
        \label{tab:hist15}
\end{table}
\begin{table}[H]
        \centering
        \begin{tabular}{|p{0.3\linewidth}|p{0.3\linewidth}|}
          \hline
          \multicolumn{2}{|c|}{Historia de usuario}\\ \hline
          \multicolumn{2}{|c|}{\textbf{Implementarla vista de \textit{dashboard}}}\\ \hline
          \textbf{Número:} 16 & \textbf{Prioridad:} Media\\ \hline
          \textbf{Estimacion:}5 días & \textbf{Iteracion:} 5\\ \hline
          \multicolumn{2}{|l|}{\textbf{Desarrollador responsable:} Pablo Palomino Gómez}\\ \hline
          \multicolumn{2}{|p{0.6\linewidth}|}{\textbf{Descripcion:} Debe crearse la vista html de inicio o panel de control. Debe contener información además de un formulario para realizar simulaciones}\\ \hline
        \end{tabular}
        \caption{Historia de usuario 16}
        \label{tab:hist16}
\end{table}
\begin{table}[H]
        \centering
        \begin{tabular}{|p{0.3\linewidth}|p{0.3\linewidth}|}
          \hline
          \multicolumn{2}{|c|}{Historia de usuario}\\ \hline
          \multicolumn{2}{|c|}{\textbf{Implementarla vista de resultados de simulación}}\\ \hline
          \textbf{Número:} 17 & \textbf{Prioridad:} Media\\ \hline
          \textbf{Estimacion:}10 días & \textbf{Iteracion:} 5\\ \hline
          \multicolumn{2}{|l|}{\textbf{Desarrollador responsable:} Pablo Palomino Gómez}\\ \hline
          \multicolumn{2}{|p{0.6\linewidth}|}{\textbf{Descripcion:} Debe crearse la vista html que muestre los resultados de realizar una simulación. Debe procesarse y mostrarse toda la información necesaria así como permitir descargar el fichero de reporte de simulación. Tambien deben implementarse los tests pertinentes}\\ \hline
        \end{tabular}
        \caption{Historia de usuario 17}
        \label{tab:hist17}
\end{table}
\textbf{Desarrollo}\\

Una vez implementada la funcionalidad de simulación, debe pensarse en añadir una persistencia al proyecto de los elementos que intervienen. Esto es necesario y primordial si se pretende crear una aplicación web para realizar simulaciones. Puesto que se desarrollará una API usando Flask~\cite{Flask} que hará la función de servidor, se ha dedicido utilizar la herramienta para gestión de base de datos SQLAlchemy.
\subsection{Persistencia con SLQAlchemy}
SQLAlchemy~\cite{SqlAl} proporciona un kit de herramientas SQL que permiten manejar bases de datos de manera eficiente. Está formado por dos componentes:
\begin{itemize}
\item \textit{Core}: Es un conjunto de herramientas de SQL que da lugar a un nivel de abstracción sobre el mismo, mediante un lenguaje que utiliza expresiones generativas en Python para expresar órdenes SQL.
\item \textit{ORM}: Se trata de un asignador relacional de objetos, es decir, permite crear una base de datos de objetos virtuales que permite manipular la información de la base de datos, a priori incompatible, como objetos utilizable por un lenguaje de programación orientada a objetos.
\end{itemize}
Mediante las consultas basadas en funciones permite ejecutar las cláusulas SQL a través de funciones y expresiones en Python. Se pueden realizar numerosas acciones como subconsultas seleccionables, insertar, actualizar, eliminar o declarar un objeto, combinaciones internas y externas sin necesidad de utilizar lenguaje SQL. El ORM permite almacenar en caché las colecciones y referencias de objetos una vez han sido cargados, dando lugar a que no sea necesario emitir SQL en cada acceso.\\

SQLAlchemy puede trabajar con bases de datos de SQLite, Postgresql, MySQL, Oracle, MS-SQL, Sybase y Firebird, entre otros.
\subsection{Modelos User y Home}
Se deben determinar los datos que se van a persistir en el sistema. Puesto que se tratará de una aplicación web con la que interactuarán usuarios, resulta interesante almacenarlos. Cada uno de estos usuarios realizará simulaciones del sistema. Como se vió anteriormente, cuando se instancia un objeto de la clase Simulation, el constructor de la misma recibe varios argumentos necesarios para llevar a cabo la simulación, de los cuales la mayoría son información acerca del hogar en el que se realizará dicha simulación (número de módulos fotovoltaicos, código de la ciudad del hogar, etc). La persistencia en base de datos de la información del hogar de cada usuario mejoraría esta situación, pues la mayoría de la información que necesita la clase Simulation sería proporcionada del hogar almacenado en base de datos de ese usuario. Por lo tanto, son necesarias dos tablas en la base de datos: \textit{Users} y \textit{Homes}, entre las cuáles existe una relación \textit{one to one}. Este tipo de relación SQL hace que ambas tablas tengan un atributo de referencia a la otra que se conoce como \textit{\textbf{foreign key}}, la cuál lo convierte en una relación bidireccional. Cada \textit{User} tendrá un \textit{Home} y viceversa. Para crear las tablas y mostrar la estructura lógica de cada una, así como sus limitaciones y atributos, se debe crear antes un \textbf{modelo de base de datos} que sea capaz de \textit{mapear} cada objeto con su tabla en la base de datos.\\

En primer lugar se debe definir la clase Base mediante \textit{sqlalchemy.ext.declarative.declarative\_base()}. Ésta clase hará el papel de superclase de cada modelo. Así, cada modelo heredado de Base corresponde con una tabla de base de datos, cuyo nombre se encuentra en el atributo \textit{\_\_tablename\_\_}. Cada objeto que se instancie la clase del modelo corresponde con un registro en base de datos. En el listado~\ref{lst:declarateModel} se muestra la sintaxis de creación de un modelo. Tras \textit{\_\_tablename\_\_} se declaran las columna que tendrá la tabla de ese modelo, indicando el tipo de dato que contiene y una serie de argumentos. Algunos de los que se han usado son:
\begin{itemize}
       \item \textbf{\textit{primary\_key}:} Cuando toma el valor \textit{True} indica que ese atributo será la clave primaria de la tabla y se usará como parámetro para instanciar a la subclase de Base.
       \item \textbf{\textit{index}:} Con \textit{True} indica que se desea que dicho atributo sea indizado y permita un rápido acceso a los registros.
       \item \textbf{\textit{unique}:} Obliga a que el atributo sea único y no permita registros con el mismo valor en esa columna de la tabla de base de datos.
       \item \textbf{\textit{backref}:} Contiene el nombre de otra tabla en la base de datos. Este argumento permite crear una relación entre ellas.
\end{itemize}
\begin{lstlisting}[language=Python,float=ht,caption={Declaración de un modelo heredado de \textit{Base}},label={lst:declarateModel}]
from sqlalchemy.ext.declarative import declarative_base

Base = declarative_base()

class <Model Name>(Base):
      __tablename__ = <Table Name>
      <attr name> = Column(<data type>, <arg>)
      ...
      ...
\end{lstlisting}
Con la información anterior ya es posible implementar los modelos para las tablas deseadas en el caso particular de este trabajo fin de grado.
\begin{itemize}
\item \textbf{Modelo User}\\
La tabla \textit{Users} hará referencia a los usuarios involucrados en el sistema. Existen una serie de atributos que serán propios de cada usuario y darán lugar a las columnas de la tabla:
\begin{itemize}
\item \textit{name}: Representa el nombre del usuario. Este dato es de tipo String y no puede ser nulo.
\item \textit{lastname}: Representa los apellidos del usuario. Toma exactamente las mismas características que el anterior.
\item \textit{email}: Como su propio nombre indica almacena el correo electrónico del usuario. No puede ser nulo y debe ser único, pues este atributo identificará a cada usuario. Además es un atributo indizado, pues se realizarán consultas a base de datos a través de él.
\item \textit{password}: Contraseña definida por el usuario para acceder a su cuenta. Puesto que la contraseña es un dato sensible, debe tratarse adecuadamente. Para ello se ha hecho uso del módulo \textit{werkzeug.security}, que se explicará en la siguiente sección de este hito en lo relativo a seguridad.
\item \textit{home}: Atributo apunta al registro de la tabla \textit{Homes} que contiene el hogar de este usuario.
\end{itemize}
Véase el listado~\ref{lst:modelUser} el cuál muestra la declaración del modelo \textit{User} heredado de base. Las funciones Integer, String, Column, relationship y declarative\_base son importadas del módulo \textbf{sqlalchemy} y permiten trabajar con abstracción sobre SQL, como se comentó anteriormente. Los métodos de clase \textit{set\_password} y \textit{check\_password} son usados para cambiar y comprobar la contraseña, respectivamente. Ambos llaman a funciones pertencientes al módulo \textit{werkzeug.security} el cuál se explicará más adelante. La función privada \textit{\_\_repr\_\_} simplemente genera un formato para mostrar un objeto Usuario, permitiendo mostrar su nombre, apellidos e email.
\begin{lstlisting}[language=Python,float=ht,caption={Modelo \textit{User}},label={lst:modelUser}]
class Users(Base):
    __tablename__ = 'usuarios'
    id = Column(Integer, primary_key=True)
    name = Column(String(100), nullable=False)
    lastname = Column(String(100), nullable=False)
    email = Column(String(100), unique=True, index=True, nullable=False)
    password_hash = Column(String(128))
    home = relationship("Homes", uselist=False, backref="Users")

    def set_password(self, password):
        self.password_hash = generate_password_hash(password)

    def check_password(self, password):
        return check_password_hash(self.password_hash, password)

    def __repr__(self):
        return (u'<{self.__class__.__name__}: {self.id}, name= {self.name} {self.lastname},' \
                ' email= {self.email}>'.format(self=self))
\end{lstlisting}
\item \textbf{Modelo Home}
La tabla \textit{Homes} representa la casa de cada usuario. Los atributos que tendrá cada registro de esta tabla en base de datos son los siguientes:
\begin{itemize}
\item \textit{city\_code}: Atributo de tipo String que hace referencia a la ciudad donde se encuentra la casa. Esto es necesario a la hora de realizar las llamadas a la API AEMET~\cite{Aemet} pues se debe proporcionar dicho código, por lo tanto este atributo no puede ser nulo.
\item \textit{pv\_modules}: Almacena el número de módulos fotovoltaicos que tendrá el hogar del usuario, por lo tanto es un atributo de tipo Integer que no puede tomar valor nulo, ya que es un dato que determina el resultado de una simulación.
\item \textit{amortization\_years\_pv}: Almacena un entero que representa el número de años en los que el usuario desea amortizar la inversión realizada en la adquisición de los módulos fotovoltaicos. El valor de este campo determina el precio en €/KWh que tendrá la energía fotovoltaica, y por ello tampoco puede ser nulo.
\item \textit{amortization\_years\_bat}: Similar al atributo anterior pero en el contexto de la obtención de la batería. En función del valor de este atributo se calcula el precio que tiene para este usuario la obtención de energía de baterías.
\item \textit{user}: Registro de la tabla \textit{Users} con el que mantiene una relación \textit{one to one} y representa el usuario propietario del hogar.
\end{itemize}
Mediante el método privado \textit{\_\_repr\_\_} se forma un formato para mostrar un registro de esta tabla, dando información acerca de la ciudad, número de módulos fotovoltaicos y id del usuario propietario.
\begin{lstlisting}[language=Python,float=ht,caption={Modelo \textit{Home}},label={lst:modelHome}]
class Homes(Base):
      __tablename__ = "homes"
      id = Column(Integer, primary_key=True)
      city_code = Column(String(100), nullable=False)
      pv_modules = Column(Integer, nullable=False)
      amortization_years_pv = Column(Integer, nullable=False)
      amortization_years_bat = Column(Integer, nullable=False)
      UserId = Column(Integer, ForeignKey('usuarios.id'), nullable=False)
      user = relationship("Users", backref="Homes")

      def __repr__(self):
          return (u'<{self.__class__.__name__}: {self.id}, city= {self.city_code}, ' \
                'pv_modules= {self.pv_modules}, ownerId= {self.UserId}>'.format(self=self))
\end{lstlisting}
\end{itemize}
Una vez definidos los modelos se puede comenzar a realizar inserciones y consultas a estas tablas en la base de datos mediante las operaciones de SQLAlchemy, pero antes se debe comprobar el correcto funcionamiento de estos modelos. Es por esto que la etapa posterior a la implementación en el ciclo de vida del software son las \textbf{pruebas}. Mediante el framework para la implementación de casos de prueba unitarios \textbf{Nose}~\cite{Nose}. Véase el Anexo B, donde se hace referencia a lo relativo a pruebas en el proyecto.\\

Para seguir la guía de buenas prácticas se crean dos bases de datos diferenciadas. Una de ella será la base de datos de \textbf{producción}, que almacenará la información real de usuario y hogares, y será usada por la aplicación web. La segunda será la base de datos de \textbf{test}, donde se realizarán las inserciones, borrados y consultas pertinentes durante el desarrollo de los casos de prueba mencionados anteriormente para garantizar un correcto funcionamiento y coherencia entre la aplicación y la persistencia de la misma.\\

Al arrancar una aplicación que hace uso de la base de datos, se debe hacer una llamada al método \textit{Base.metadata.create\_all} proporcionando como argumento el objeto instanciado de \textit{sqlalchemy.engine.base.Engine}. Esto es necesario ya que permite crear las tablas en la base de datos porporcionada si estas no existen, y si existe, se crearán las nuevas tablas en caso de haberlas y no se eliminan los datos que existen. A partir de dicha llamada, cada objeto que se instancia de cada uno de los modelos se corresponde con un registro de su tabla correspondiente. Después de esto la base de datos estaría totalmente operable desde la aplicación. El ORM de SQLAlchemy permite realizar búsquedas relaccionadas con los objetos instanciados del modelo \textit{Base} y sus tablas relacionadas. Las operaciones se realizan en sesiones, que terminan con un \textit{commit} que persiste los cambios realizados en la sesión en la base de datos. En el listado~\ref{lst:consultaUser} se muestra una consulta para obtener el primer usuario de la tabla \textit{Users}. Nótese que el formato de representación devuelto es el definido en la clase del modelo~\ref{lst:modelUser} mediante el método \textit{\_\_repr\_\_}.
\begin{lstlisting}[language=Python,float=ht,numbers=none,caption={Consulta para obtener el primer \textit{User}},label={lst:consultaUser}]
>>> db.session.query(Users).first()
<Users: 1, name= Pablo Palomino Gomez, email= pablo@eoptimizer.com>
\end{lstlisting}

Para integrar la persistencia que se ha incorporado al funcionamiento actual, debe refactorizarse la clase Simulation. El constructor pasa de recibir un argumento por atributo de clase a recibir únicamente tres atributos: home, user y date. Toda la información necesaria para una simulación que antes se obtenía a través del fichero de constantes del proyecto (y no entendía de usuarios) ahora se puede obtener de los objetos \textit{User} y \textit{Home} del usuario que realiza la simulación. El tercer parámetro (date) hace referencia a un objeto de tipo datetime correspondiente al día en el que se realiza la simulación.\\

A partir de esta integración, se permite realizar simulaciones dependientes de un usuario, que obtendrá un valor totalmente distinto a si otro usuario realiza la simulación el mismo día, pues cada uno de ellos tiene unos parámetros en su hogar que determinan el resultado, como son el número de módulos fotovoltaicos, la ciudad donde reside o el precio al que obtiene las energías fotovoltaica y de batería, ya que cada usuario define el periodo en el que desea amortizar lo invertido en cada fuente mediante ganancias de esa fuente.

\subsection{Creación de una aplicación web con Flask}
El siguiente paso tras la integración de persistencia y la implementación de la lógica o \textit{backend} es crear una aplicación web siguiendo la arquitectura \textbf{cliente - servidor}~\cite{Goer04}. En esta arquitectura, cada una de las máquinas que realiza una demanda de información al sistema toma el rol de cliente, y la que responde a estas demandas toma el rol de servidor~\ref{fig:client-server}. Esto permitirá que exista un servidor el cuál realiza simulaciones a petición de clientes, de los que previamente se ha almacenado la información necesaria en base de datos mediante un registro.
\begin{figure}[!h]
            \centering
            \includegraphics[width=7cm]{figs/client-server.png}
            \caption{Esquema Cliente - Servidor}
            \label{fig:client-server}
\end{figure}
El nombre que tomará dicha aplicación web será \textbf{eOptimizer}, haciendo referencia a su objetivo principal: realizar simulaciones por medio de una optimización. En la figura~\ref{fig:logo} se muestra el logo creado para la aplicación. Dicho logo ha sido diseñado por el propio alumno y se inspira en la eficiencia energética que se consigue mediante el sistema que se propone.
\begin{figure}[!h]
            \centering
            \includegraphics[width=3cm]{figs/logo.png}
            \caption{Logo de eOptimizer}
            \label{fig:logo}
\end{figure}
Antes de entrar en la implementación de la aplicación debe definirse correctamente el funcionamiento que ha de tener. Esto implica determinar el flujo posible de interación con la aplicación, algo que a la hora de implementar captura de errores y restricciones facilitará mucho la complejidad. Para ello ha de construirse un diagrama de flujo. Este tipo de diagrama describe un sistema informático, y tiene como objetivo planificar, estudiar y diseñar procesos complejos o algoritmos. De entre todas las funcionalidades que pueden implementar los diagramas de flujo, deben mencionarse:
\begin{itemize}
\item Mostrar la ejecución del código de un software.
\item Facilitar la comprensión de la estructura y funcionalidad de una aplicación web.
\item Explicar visualmente cómo un usuario puede navegar por una página web.
\end{itemize}
Para crear un diagrama de flujo debe tenerse en cuenta el alcance del sistema. En el caso de este trabajo fin de grado, un usuario (rol de cliente) interactuará con la aplicación web (servidor) con el objetivo de realizar una \textbf{simulación} del consumo eléctrico de un día determinado que habría tenido su hogar mediante el sistema propuesto, comparándolo frente a su situación actual, por tanto la entrada al diagrama de flujo debe ser el hecho de acceder a la aplicación web, y la salida será la información pertinente a la simulación realizada.\\

Lo primero que debe hacer un usuario es identificarse, pues en el servidor debe tenerse en cuenta el usuario de la base de datos que está interactuando con el sistema para realizar la simulación, ya que necesita su información asociada. Por ello, el usuario deberá introducir sus credenciales para poder acceder. Si un usuario nunca antes ha accedido a eOptimizer, deberá registrarse. Una vez haya introducido la información de usuario para su registro, esta se insertará en base de datos y si todo va correctamente, el siguiente paso es introducir la información asociada a su hogar, que será empleada para realizar sus simulaciones. Cuando se haya introducido se insertará en base de datos, teniendo persistida toda la información necesaria del usuario en curso, por lo que se permitiría el acceso al índice o \textit{dashboard} de la web. Desde allí un usuario podrá realizar simulaciones, introduciendo los datos necesarios para la misma, como son la fecha que desea simular y el fichero de consumo de Endesa que hubo en su hogar ese día. Si esta información es correcta, el \textit{backend} realizará la optimización habiendo obtenido previamente la información de las APIs AEMET y Esios relativa al día de simulación, mostrando los resultados obtenidos al cliente. En la figura~\ref{fig:diagrama-flujo} se muestra el diagrama de flujo de la aplicación que permite visualizar todo el funcionamiento de manera mas sencilla.
\begin{figure}[!h]
            \centering
            \includegraphics[width=15cm]{figs/diagrama_flujo.png}
            \caption{Diagrama de flujo de eOptimizer}
            \label{fig:diagrama-flujo}
\end{figure}

\section{Migración de la aplicación a la nube mediante una \textit{IaaS}}
\textbf{Historias de Usuario}\\

\begin{table}[H]
        \centering
        \begin{tabular}{|p{0.3\linewidth}|p{0.3\linewidth}|}
          \hline
          \multicolumn{2}{|c|}{Historia de usuario}\\ \hline
          \multicolumn{2}{|c|}{\textbf{Identificar los recursos que deben migrarse}}\\ \hline
          \textbf{Número:} 18 & \textbf{Prioridad:} Baja\\ \hline
          \textbf{Estimacion:}4 horas & \textbf{Iteracion:} 6\\ \hline
          \multicolumn{2}{|l|}{\textbf{Desarrollador responsable:} Pablo Palomino Gómez}\\ \hline
          \multicolumn{2}{|p{0.6\linewidth}|}{\textbf{Descripcion:}Identificar los recursos del sistema que pueden migrarse a la nube y elegir proveedores para ello}\\ \hline
        \end{tabular}
        \caption{Historia de usuario 18}
        \label{tab:hist18}
\end{table}
\begin{table}[H]
        \centering
        \begin{tabular}{|p{0.3\linewidth}|p{0.3\linewidth}|}
          \hline
          \multicolumn{2}{|c|}{Historia de usuario}\\ \hline
          \multicolumn{2}{|c|}{\textbf{Migrar los recursos elegidos y hacer accesible la aplicación desde Internet}}\\ \hline
          \textbf{Número:} 19 & \textbf{Prioridad:} Baja\\ \hline
          \textbf{Estimacion:}10 días & \textbf{Iteracion:} 6\\ \hline
          \multicolumn{2}{|l|}{\textbf{Desarrollador responsable:} Pablo Palomino Gómez}\\ \hline
          \multicolumn{2}{|p{0.6\linewidth}|}{\textbf{Descripcion:} Migrar cada uno de los recursos elegidos a la nube y hacer funcional la aplicación tras ello. Perimitr el acceso desde Internet al sistema y programar su ejecución constante}\\ \hline
        \end{tabular}
        \caption{Historia de usuario 19}
        \label{tab:hist19}
\end{table}
\textbf{Desarrollo}\\

La computación en la nube (\textit{cloud computing}) permite ubicuidad en el acceso a un conjunto de recursos compartido, como pueden ser servidores, bases de datos, contenedores, almacenamiento, servicios, etc. Estos recursos pueden ser aprovisionados o liberados con facilidad y bajo demanda, teniendo una facturación por uso y siendo accedidos a través de la red, lo que da lugar a que esta tendencia cada vez esté mas en auge. Los recursos ofrecidos al usuario son ilimitados, teniendo a su disposición arquitecturas de computación realmente complejas y modernas sin un coste desorbitado.\\
Antes del \textit{cloud computing} el hecho de crear un servidor que funcionase 24 horas al día, 7 días a la semana era algo costoso, ya que no solo implica la inversión de la máquina si no que conlleva unos costes de mantenimiento (alimentación y red constantes, mejora de componentes desfasados a lo largos del tiempo, mantenimiento técnico, etc). Con esta nueva tendencia, es posible contratar el servicio deseado facturando únicamente por uso sin necesidad de todos los gastos anteriores.\\

El primer paso es analizar que recursos en la nube son necesarios para la aplicación web de este \gls{TFG}. Existen un gran tipo de recursos disponibles de distintos proveedores:
\begin{itemize}
\item Instancias
\item Bases de Datos SQL
\item Bases de Datos NoSQL
\item Almacenamiento en la nube
\item IP virtuales SSL
\end{itemize}

Por un lado es interesante que la persistencia del sistema cuente con un recurso propio, para lo que sería necesario una base de datos SQL, pues se utiliza el framework SQLAlchemy que trabaja sobre bases de datos SQL. El proveedor elegido para la instancia de este recurso ha sido la plataforma \textbf{IBM Cloud}~\cite{IMB} (Figura~\ref{fig:IBMCloud}).
\begin{figure}[H]
            \centering
            \includegraphics[width=5cm]{figs/ibm_logo.png}
            \caption{Logo de la plataforma IBM Cloud}
            \label{fig:IBMCloud}
\end{figure}

IBM Cloud combina una plataforma como servicio (\gls{PaaS}) y una infraestructura como servicio (\gls{IaaS}) poniendo a disposición de desarrolladores un catálogo de recursos muy amplio. Ha sido elegido debido a que numerosos recursos de dicho catálogo cuentan con una opción \textit{lite} con capacidades limitadas totalmente gratuita, idónea para pequeños desarrolladores que se inician en la experiencia del \textit{cloud computing}. En la Figura \ref{fig:catalogoIBM} se muestran los recursos \textit{lite} en la categoría de bases de datos que IBM cloud ofrece.
\begin{figure}[H]
            \centering
            \includegraphics[width=14cm]{figs/db_ibm.png}
            \caption{Catálogo de Bases de Datos en IBM Cloud}
            \label{fig:catalogoIBM}
\end{figure}
Puesto que es necesario una base de datos SQL el recurso elegido ha sido \textbf{Db2}. Este servicio ofrece una base de datos SQL de 200 MB en su versión gratuita con copias de seguridad y permitiendo cinco conexiones simultáneas, algo más que suficiente para lo necesario en este \gls{TFG}. Una vez adquirido el recurso se permite el acceso a una consola de gestión de la base de datos que permite ver información acerca de las conexiones recibidas, información de la base de datos como tablas y registros existentes, almacenamiento restante, etc.\\El primer paso tras la obtención del recurso es generar las credenciales del servicio. Estas credenciales son proporcionadas en formato \gls{JSON} y contienen información como hostname, usuario, puerto, ssldsn, uri, etc. Para realizar la integración de esta base de datos de producción remota en la aplicación web antes debe instalarse en la máquina del servidor el controlador ibm\_db\_sa que permite las conexiones entre db2 y el framework SQLAlchemy.\\El siguiente paso es actualizar en el fichero de configuración de la aplicación Flask para producción la constante \textbf{SQLALCHEMY\_DATABASE\_URI} (listado~\ref{lst:Db2URI}) que actualmente se encuentra apuntando a una base de datos local sqlite. Su nuevo valor será el URI generado anteriormente en las credenciales del servicio. El URI es un identificador de recursos uniforme, que permite apuntar inequívocamente a la base de datos de IBM, pues contiene toda la información necesaria (usuario, key, puerto, host, etc). Cada uno de estos valores se ha añadido al fichero de constantes del proyecto (\textit{project\_constants}) excepto la contraseña, que por medida de seguridad es obtenida de las variables de entorno. Cuando la instancia de la aplicación Flask se inicia con la nueva configuración, la conexión con la base de datos de IBM se realiza y las operaciones son efectuadas en ella. Como anotación, la base de datos de test sigue siendo utilizada en local, pues no tiene sentido instanciar en la nube una base de datos con este fin ya que tras ejecutar los test es limpiada y no contiene información relevante.\\
\begin{lstlisting}[language=Python,float=ht,numbers=none,caption={URI de Db2 para SQLAlchemy en \textit{prod\_config}},label={lst:Db2URI}]
SQLALCHEMY_DATABASE_URI = 'db2://{}:{}@{}:{}/{}'.format(
                              const.IBM_USER,
                              os.environ['DB2_EOPTIMIZER_KEY'],
                              const.IBM_HOSTNAME,
                              const.IBM_PORT,
                              const.IBM_DB
                           )
\end{lstlisting}

Por otro lado, la aplicación web de este trabajo necesita de un servidor que ha de estar funcionando constantemente, para el cuál sería necesario una instancia. El proveedor elegido para este recurso ha sido \textbf{Amazon Web Services}~\cite{AWS} (Figura~\ref{fig:AWS}).
\begin{figure}[H]
            \centering
            \includegraphics[width=4cm]{figs/aws_logo.png}
            \caption{Logo de Amazon Web Services}
            \label{fig:AWS}
\end{figure}

\gls{AWS} proporciona una infraestructura como servicio (\gls{IaaS}) para empresas y desarrolladores en forma de servicios web. Es una infraestructura escalable, segura, de bajo costo y muy flexible lo que lo convierte en uno de los principales proveedores en el mundo del \textit{cloud computing}. La Universidad de Castilla-La Mancha cuenta con un convenio que permite acceso a la iniciativa AWS Educate, mediante la cuál Amazon proporciona acceso a los recursos de Amazon Web Services a los estudiantes IT, entre otras ventajas.\\
De entre todos los recursos existentes en el catálogo de \gls{AWS}, para este \gls{TFG} se hará uso de una instancia \gls{EC2}, la cuál proporciona capacidad de computación escalable en la nube, eliminando la necesidad de invertir en hardware. Puesto que \gls{AWS} es una infraestructura escalable, existen numerosas configuraciones a la hora de crear una instancia \gls{EC2}. En la Tabla~\ref{tab:EC2instance} se muestra la configuración de la instancia creada para la aplicación web de este \gls{TFG}. En la Figura~\ref{fig:AWSControlPanel} se muestra el panel de control de instancias de \gls{AWS}, con la información referente a la instancia creada.
\begin{table}[hp]
        \centering
        \begin{tabular}{|l|c|}
                \hline
                Tipo de instancia & t2.micro \\ \hline
                Sistema Operativo & Ubuntu Server 18.04 LTS \\ \hline
                Memoria & 1 GB \\ \hline
                Disco & 8 GB SSD \\ \hline
             \end{tabular}
        \caption{Instancia EC2 creada en AWS}
        \label{tab:EC2instance}
\end{table}

\begin{figure}[H]
            \centering
            \includegraphics[width=17cm]{figs/aws_control_panel.png}
            \caption{Panel de control de instancias AWS}
            \label{fig:AWSControlPanel}
\end{figure}

Una vez seleccionada la instancia, el siguiente paso es generar la claves ssh. Esto permitirá el acceso remoto mediante el protocolo ssh a la máquina, la cuál es preparada para ejecutar la aplicación web mediante la instalación de las dependencias necesarias. Cuando el servidor esté corriendo, este será accesible a través de la IP pública de la instancia \gls{EC2} y el puerto seleccionado para la aplicación. Dicha IP es una \textbf{IP elástica} proporcionada por \gls{AWS}, las cuáles tienen la capacidad de enmascarar los errores producidos en una instacia reasignando rápidamente la dirección con otra instancia similar del usuario \gls{AWS}. Para permitir la ejecución constante de la aplicación web se han creado dos tareas \textbf{cron}. Cron es un administrador regular de procesos -demonios- que permite pogramar tareas disparadas por tiempo, fecha o procesos, permitiendo automatizar las tareas de gestión de un servidor web como ponerlo en funcionamiento, pararlo, limpiar ficheros basura generados, etc. Se ha decidido crear dos script bash, el primero para ejecutar la aplicación web y el segundo para parar su ejecución y limpiar los archivos generados (copias de los reportes de simulación realizados). Las tareas cron creadas para el mantenimiento del servidor han sido las siguientes:
\begin{itemize}
\item Cada semana la aplicación web es parada durante la madrugada, y arrancada nuevamente tras la limpieza de basura generada durante la semana.
\item Cuando la instancia \gls{EC2} se inicia, la aplicación se ejecuta. Esto es debido a posibles errores en la instancia \gls{EC2} que requieran su reinicio, permitiendo así que esto no afecte al servidor.
\end{itemize}

Tras esta iteración la aplicación web eOptimizer se encuentra en constante funcionamiento accesible mediante el socket (dirección IP y puerto) mostrado a continuación, cuyos dos recursos se encuentran alojados en la nube, cada uno en un proveedor distinto y conectados entre sí.\\

\centerline{\url{http://3.213.79.178:5000/}}

